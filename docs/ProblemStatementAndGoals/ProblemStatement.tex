\documentclass[12pt,letterpaper]{article}
\usepackage[utf8]{inputenc}
\usepackage[margin=1in]{geometry}

\usepackage{tabularx}
\usepackage{booktabs}
\usepackage[table, dvipsnames]{xcolor}
\usepackage{longtable}

\title{Problem Statement and Goals\\\progname}

\author{\authname}

\date{}

%% Comments

\usepackage{color}

\newif\ifcomments\commentstrue %displays comments
%\newif\ifcomments\commentsfalse %so that comments do not display

\ifcomments
\newcommand{\authornote}[3]{\textcolor{#1}{[#3 ---#2]}}
\newcommand{\todo}[1]{\textcolor{red}{[TODO: #1]}}
\else
\newcommand{\authornote}[3]{}
\newcommand{\todo}[1]{}
\fi

\newcommand{\wss}[1]{\authornote{blue}{SS}{#1}} 
\newcommand{\plt}[1]{\authornote{magenta}{TPLT}{#1}} %For explanation of the template
\newcommand{\an}[1]{\authornote{cyan}{Author}{#1}}

%% Common Parts

\newcommand{\progname}{REVITALIZE} % PUT YOUR PROGRAM NAME HERE
\newcommand{\authname}{Team 13, REVITALIZE
\\ Bill Nguyen
\\ Syed Bokhari
\\ Hasan Kibria
\\ Youssef Dahab
\\ Logan Brown
\\ Mahmoud Anklis} % AUTHOR NAMES                  

\usepackage{hyperref}
    \hypersetup{colorlinks=true, linkcolor=blue, citecolor=blue, filecolor=blue,
                urlcolor=blue, unicode=false}
    \urlstyle{same}
                                


\begin{document}

\maketitle

\begin{table}[hp]
\caption{Revision History} \label{TblRevisionHistory}
\begin{tabularx}{\textwidth}{llX}
\toprule
\textbf{Date} & \textbf{Developer(s)} & \textbf{Change}\\
\midrule
Date1 & Name(s) & Description of changes\\
Date2 & Name(s) & Description of changes\\
... & ... & ...\\
\bottomrule
\end{tabularx}
\end{table}

\section{Problem Statement}

\wss{You should check your problem statement with the
\href{https://github.com/smiths/capTemplate/blob/main/docs/Checklists/ProbState-Checklist.pdf}
{problem statement checklist}.}
\wss{You can change the section headings, as long as you include the required information.}

\subsection{Problem}

\subsection{Inputs and Outputs}

\wss{Characterize the problem in terms of ``high level'' inputs and outputs.  
Use abstraction so that you can avoid details.}

\subsection{Stakeholders}

\par
Primary Stakeholders: Adults/teenagers who want to improve and keep track of their overall health and wellness via an easy to use, all in one application.

\noindent
\\
Secondary Stakeholders: Individuals who may not use the application directly for themselves or not directly involved with the use of the application but have an indirect benefit. An example are personal trainers can use this application to keep track of workouts, sleep, overall health etc. of their clients.

\noindent
\\
Facilitating Stakeholders: All team 13 (REVITALIZE) members building application, Dr. Spencer Smith, 4G06 TAs, Project Supervisor.

\subsection{Environment}

Software application, where if web application is created, will be available for all devices, but if mobile application is created will be only available for android devices.

\section{Goals}

\begin{longtable}{ |p{3cm}|p{5cm}|p{5cm}| }
 \caption{Product Goals} \\ 
\toprule 
\textbf{Goal} &\textbf{ Explanation} & \textbf{Reasoning}\\
 \midrule
 Ease of Use & The user interface and user interaction of application should be innate and simplistic so users of all levels can smoothly identify all application's features/components and instructions shown to them.    & A simplistic product will help reduce confusion and time spent figuring out how to use product, which can lead to more time spent using the product more effectively to help improve health and wellness, and make overall experience more enjoyable.  \\
 \hline
 Accurate Data & User data for all components of application such as recipe finder, sleep tracker etc. is outputted and maintained precisely based on the given inputs and communication between user and application. & Must provide reliable and precise data, in order to ensure trust to users that our product works as intended, in order for users to accurately track overall health and wellness progress. Also providing accurate data, that can match or exceed rival products, can provide a competitive advantage and can make it easier to penetrate the intended market.\\
 \hline
 Secure & All private user data, should be protected and application should prevent any data leakages from occurring. User data should only be available for the user unless, user specifies otherwise.  & In society, there are many hacking attempts, and hacking has become a profitable business where the victims are innocent users. So by making application more secure by creating high-level authentication and testing our product thoroughly, it can help ensure more trust to our users. \\
 \hline
 Customizable & The application should be flexible and customizable to cater to the user's personal health and limitations. The Recipe Finder, Workout Planner and Sleep Scheduler should have preset values that can be altered to the user's preferences.  & Each individual has a varying body types, stengths and weaknesses. The Recipe Finder should have options that consider dietary restrictions and calorie goals. The Workout planner should have a large variety of excercises to choose from for each muscle group. The sleep tracker should be able to adjust the sleep timer to allow for planning within the user's schedule. \\
 \hline
 Progress Tracking & User interactions with the app should be logged and tracked in a diary format to ensure that the user is able to better understand their daily habits & When making a lifestyle change, it is important to keep track of progress to ensure that daily, weekly and monthly targets are met. It is essential for the user to be able to reflect on their habits and routine and adjust to better aid in their journey.   \\
 \hline
 Digestable Visualization of Big Data & The data accumulated throughout the use of the app should be easily accessable in a digestable format to help the user better understand progress and keep track of their longterm health  & User's can have the option to view their progress in terms of long term data sets that can help to visualize progress. A visual representation can be extemely beneficial to understanding the change in habits over a long period of time.    \\
 \hline
\end{longtable}

\section{Stretch Goals}

\begin{longtable}{ |p{3cm}|p{5cm}|p{5cm}| }
 \caption{Product Goals} \\ 
\toprule 
\textbf{Stretch Goal} &\textbf{ Explanation} & \textbf{Reasoning}\\
 \midrule
 Personal Traine Integration & User diet, workout routine and sleep schedule can be set by personal trainer. The trainer can also view user progress and adjust schedule & Integration with a professional can help ease the lifestyle transition and can be used as a source of motivation to stay on track  \\
 \hline
 Sleep data predictions & User sleep data points can be used to extrapolate user health conditions & Sleep is a fundemental indicator of overall health and wellness. The sleep data can explain many issues the user is facing during the day. \\
 \hline

\end{longtable}

\end{document}