\documentclass[12pt, titlepage]{article}

\usepackage{booktabs}
\usepackage{tabularx}
\usepackage{hyperref}
\hypersetup{
    colorlinks,
    citecolor=blue,
    filecolor=black,
    linkcolor=red,
    urlcolor=blue
}
\usepackage[round]{natbib}

%% Comments

\usepackage{color}

\newif\ifcomments\commentstrue %displays comments
%\newif\ifcomments\commentsfalse %so that comments do not display

\ifcomments
\newcommand{\authornote}[3]{\textcolor{#1}{[#3 ---#2]}}
\newcommand{\todo}[1]{\textcolor{red}{[TODO: #1]}}
\else
\newcommand{\authornote}[3]{}
\newcommand{\todo}[1]{}
\fi

\newcommand{\wss}[1]{\authornote{blue}{SS}{#1}} 
\newcommand{\plt}[1]{\authornote{magenta}{TPLT}{#1}} %For explanation of the template
\newcommand{\an}[1]{\authornote{cyan}{Author}{#1}}

%% Common Parts

\newcommand{\progname}{REVITALIZE} % PUT YOUR PROGRAM NAME HERE
\newcommand{\authname}{Team 13, REVITALIZE
\\ Bill Nguyen
\\ Syed Bokhari
\\ Hasan Kibria
\\ Youssef Dahab
\\ Logan Brown
\\ Mahmoud Anklis} % AUTHOR NAMES                  

\usepackage{hyperref}
    \hypersetup{colorlinks=true, linkcolor=blue, citecolor=blue, filecolor=blue,
                urlcolor=blue, unicode=false}
    \urlstyle{same}
                                


\begin{document}

\title{Project Title: System Verification and Validation Plan for \progname{}} 
\author{\authname}
\date{\today}
	
\maketitle

\pagenumbering{roman}

\section{Revision History}

\begin{tabularx}{\textwidth}{p{3cm}p{2cm}X}
\toprule {\bf Date} & {\bf Version} & {\bf Notes}\\
\midrule
October 31st, 2022 & Bill Nguyen & Added Functional Requirements Tests for Login Page\\
October 31st, 2022 & Bill Nguyen & Added Non Functional Requirements Tests for Look and Feel, Usability and Performance\\
October 31st, 2022 & Bill Nguyen & Added 3 Questions to Usability Survey\\
\bottomrule
\end{tabularx}

\newpage

\tableofcontents

\listoftables
\wss{Remove this section if it isn't needed}

\listoffigures
\wss{Remove this section if it isn't needed}

\newpage

\section{Symbols, Abbreviations and Acronyms}

\renewcommand{\arraystretch}{1.2}
\begin{tabular}{l l} 
  \toprule		
  \textbf{symbol} & \textbf{description}\\
  \midrule 
  T & Test\\
  \bottomrule
\end{tabular}\\

\wss{symbols, abbreviations or acronyms -- you can simply reference the SRS
  \citep{SRS} tables, if appropriate}

\newpage

\pagenumbering{arabic}

This document ... \wss{provide an introductory blurb and roadmap of the
  Verification and Validation plan}

\section{General Information}

\subsection{Summary}

\wss{Say what software is being tested.  Give its name and a brief overview of
  its general functions.}

\subsection{Objectives}

\wss{State what is intended to be accomplished.  The objective will be around
  the qualities that are most important for your project.  You might have
  something like: ``build confidence in the software correctness,''
  ``demonstrate adequate usability.'' etc.  You won't list all of the qualities,
  just those that are most important.}

\subsection{Relevant Documentation}

\wss{Reference relevant documentation.  This will definitely include your SRS
  and your other project documents (MG, MIS, etc).  You can include these even
  before they are written, since by the time the project is done, they will be
  written.}

\citet{SRS}

\section{Plan}

\wss{Introduce this section.   You can provide a roadmap of the sections to
  come.}

\subsection{Verification and Validation Team}

\wss{You, your classmates and the course instructor.  Maybe your supervisor.
  You shoud do more than list names.  You should say what each person's role is
  for the project.  A table is a good way to summarize this information.}

\subsection{SRS Verification Plan}

\wss{List any approaches you intend to use for SRS verification.  This may just
  be ad hoc feedback from reviewers, like your classmates, or you may have
  something more rigorous/systematic in mind..}

\wss{Remember you have an SRS checklist}

\subsection{Design Verification Plan}

\wss{Plans for design verification}

\wss{The review will include reviews by your classmates}

\wss{Remember you have MG and MIS checklists}

\subsection{Implementation Verification Plan}

\wss{You should at least point to the tests listed in this document and the unit
  testing plan.}

\wss{In this section you would also give any details of any plans for static verification of
  the implementation.  Potential techniques include code walkthroughs, code
  inspection, static analyzers, etc.}

\subsection{Automated Testing and Verification Tools}

\wss{What tools are you using for automated testing.  Likely a unit testing
  framework and maybe a profiling tool, like ValGrind.  Other possible tools
  include a static analyzer, make, continuous integration tools, test coverage
  tools, etc.  Explain your plans for summarizing code coverage metrics.
  Linters are another important class of tools.  For the programming language
  you select, you should look at the available linters.  There may also be tools
  that verify that coding standards have been respected, like flake9 for
  Python.}

\wss{The details of this section will likely evolve as you get closer to the
  implementation.}

\subsection{Software Validation Plan}

\wss{If there is any external data that can be used for validation, you should
  point to it here.  If there are no plans for validation, you should state that
  here.}

\section{System Test Description}
	
\subsection{Tests for Functional Requirements}

Subsections of the requirements will be divided into the events from our SRS, which are Login Page, Sign up Page, Main Page, Diet Section, Workout Section and Rest Section. There will be 1 test per functional requirement, and will follow the same order as functional requirements in SRS (ex. FR1 in VnV plan is the test for FR1 in SRS).

\subsubsection{Login Page Testing}

Testing all functional requirements for login page of REVITALIZE. (Refer to BE1 in SRS)

\begin{enumerate}

\item{FR-LP-1\\}

Control: Manual
					
Initial State: Loading stage of the login page
					
Input: An event that loads the login page
					
Output: Login page is displayed with all necessary components

Test Case Derivation: Request is made to load login page
					
How test will be performed: Tester will open REVITALIZE application and login page should be displayed
					
\item{FR-LP-2\\}

Control: Manual
					
Initial State: Login page is displayed with username textbox
					
Input: Enter username information in textbox
					
Output: Username information entered is displayed in textbox

Test Case Derivation: User can enter information in username textbox

How test will be performed: Tester will enter information in username textbox and checks if textbox displays what the tester entered.

\item{FR-LP-3\\}

Control: Manual
					
Initial State: Login page is displayed with password textbox
					
Input: Enter password information in textbox
					
Output: password information entered is displayed in textbox via hidden text

Test Case Derivation: User can enter information in password textbox

How test will be performed: Tester will enter information in password textbox and checks if textbox displays what the tester entered via hidden text.

\item{FR-LP-4\\}

Control: Manual
					
Initial State: Login page is displayed with login button
					
Input: Click login button
					
Output: Intended events occurs. Refer to FR9

Test Case Derivation: User clicks login button and a request is made based on username and password text-boxes

How test will be performed: Tester will click on login button and check if request is made correctly

\item{FR5\\}

Control: Manual
					
Initial State: Login page is displayed with forgot password button
					
Input: Click forgot password button
					
Output: Display forgot password screen with textbox to enter email

Test Case Derivation: User clicks forgot password button and request is made to display forgot password screen with textbox to enter email

How test will be performed: Tester will click on forgot password button and checks if forgot password screen is displayed with textbox to enter email

\item{FR-LP-6\\}

Control: Manual
					
Initial State: Login page is displayed with stay logged in checkbox that is empty
					
Input: Click stay logged in checkbox
					
Output: Display a check-mark in the stay logged in checkbox if checkbox is empty, else if checkbox contains check-mark already it will then display an empty checkbox

Test Case Derivation: User clicks on stay logged in checkbox and displays appropriate action

How test will be performed: Tester will click on checkbox and checks to see if check-mark is displayed if checkbox was empty and if an empty checkbox appears if checkbox contained a check-mark

\item{FR-LP-7\\}

Control: Manual
					
Initial State: Loading stage of REVITALIZE where previous state had stay logged in checkbox checked
					
Input: An event that loads REVITALIZE
					
Output: Display main page, with same data from previous state of main page

Test Case Derivation: User can load REVITALIZE and main page is displayed with same data as the previous time user opened REVITALIZE main page

How test will be performed: Tester will check stay logged in checkbox go to main page, leave REVITALIZE, reopen REVITALIZE and check whether same data from main page is the same from the last time tester opened main page

\item{FR-LP-8\\}

Control: Manual
					
Initial State: Login page is displayed with sign up button
					
Input: Click sign up button
					
Output: Loads and displays sign up page

Test Case Derivation: User can click sign up button which loads and displays sign up page

How test will be performed: Tester will click on sign up button and checks if sign up page is displayed

\item{FR-LP-9\\}

Control: Manual
					
Initial State: Login page is displayed with inputted information in username and password text-boxes
					
Input: Click login button
					
Output: if failure state, display an invalid password or username banner, else if success state, load and display main page

Test Case Derivation: User clicks login button and request is made based on username and password, and will proceed to main page only if username and password are valid

How test will be performed: Tester will click on login button and test for scenarios when login should be successful and when login should fail


\end{enumerate}

\subsubsection{Signup Page Testing}

Testing all functional requirements for sign page of REVITALIZE. (Refer to BE2 in SRS)

\begin{enumerate}

\item{FR-SP-1\\}

Control: Manual
					
Initial State: Signup page is displayed with username textbox
					
Input: Enter username information in textbox
					
Output: Username information entered is displayed in textbox

Test Case Derivation: User can enter information in username textbox

How test will be performed: Tester will enter information in username textbox and checks if textbox displays what the tester entered
					
\item{FR-SP-2 \\}

Control: Manual
					
Initial State: Signup page is displayed with password textbox
					
Input: Enter password information in textbox
					
Output: Password information entered is displayed in textbox via hidden text

Test Case Derivation: User can enter information in password textbox

How test will be performed: Tester will enter information in password textbox and checks if textbox displays what the tester entered via hidden text

\item{FR-SP-3\\}

Control: Manual
					
Initial State: signup page is displayed with email textbox
					
Input: Enter email information in textbox
					
Output: Email information entered is displayed in textbox

Test Case Derivation: User can enter information in email textbox

How test will be performed: Tester will enter information in email textbox and checks if textbox displays what the tester entered

\item{FR-SP-4\\}

Control: Manual
					
Initial State: Signup page is displayed with signup button
					
Input: Click signup button
					
Output: Intended events occurs. Refer to FR14

Test Case Derivation: User clicks signup button and a request is made based on username, password and email text-boxes

How test will be performed: Tester will click on signup button and check if request is made correctly

\item{FR-SP-5\\}

Control: Manual
					
Initial State: Signup page is displayed with inputted information in username and password text-boxes
					
Input: Click signup button
					
Output: if failure state, display an invalid username, password or email banner, else if success state, load and display login page

Test Case Derivation: User clicks signup button and request is made based on username, password and email, and will proceed to login page only if username, password and email are valid

How test will be performed: Tester will click on signup button and test for scenarios when signup should be successful and when signup should fail


\end{enumerate}

\subsubsection{Main Page Testing}

Testing all functional requirements for main page of REVITALIZE. (Refer to BE3 in SRS)

\begin{enumerate}

\item{FR-MP-1\\}

Control: Manual
					
Initial State: Main page is displayed with calender of current date
					
Input: An event that loads the main page
					
Output: Main page is displayed with all necessary components

Test Case Derivation: Request is made to load main page

How test will be performed: Tester will successfully login to the REVITALIZE application and will visually check if the correct calender data is loaded
					
\item{FR-MP-2\\}

Control: Manual
					
Initial State: Main page is displayed with previous day and next day buttons
					
Input: An event that loads the main page and the previous day and next day  buttons are clicked
					
Output: Main page is displayed with previous day and next day buttons. Once the next day button is clicked, the calender refreshes to a new calender with the next day. Once the preivous day button is clicked, the calender refreshed to a new calender with the previous day

Test Case Derivation: Request is made to load main page and the previous day and next day buttons are clicked

How test will be performed: Tester will successfully login to the REVITALIZE application and will visually check if the previous day and next day buttons are visible. The tester will click the previous day button and visually check if the calender is updated to the date of the previous day. The tester will click the next day button and visually check if the calender is updated to the date of the next day

\item{FR-MP-3\\}

Control: Manual
					
Initial State: Each interaction after leaving the main page must have a visible back button
					
Input: An event that loads the next user interface after leaving the main page and the back button is clicked
					
Output: The next user interface after leaving the main page iis displayed with a back button. Once the back button is clicked, the main page is loaded

Test Case Derivation: Request is to leave the main page and the back button is clicked

How test will be performed: Tester will leave the main page by selecting any of the options on the page. The tester will visually check if a back button is visible on every page that is entered through the main page interaction. The tester will click the back button and visually check if the current page is closed and the main page is loaded. The tester will repeat this process with every page that is  loaded from clicking an interaction from the main page
					
\item{FR-MP-4\\}

Control: Manual
					
Initial State: Main page is displayed with Diet, Exercise and Rest buttons available to click
					
Input: An event that loads the main page and the Diet, Exercise and Rest buttons are clicked

Output: Main page is displayed with Diet, Exercise and Rest buttons. If the Diet button is clicked, the Diet interface is loaded. If the Exercise button is clicked, the Exercise interface is loaded. If the Rest button is clicked, the Rest interface is loaded

Test Case Derivation: Request is made to load main page and the Diet, Exercise and Rest buttons are clicked

How test will be performed: Tester will successfully login to the REVITALIZE application and will visually check if the Diet, Exercise and Rest buttons are visible. The tester will click the Diet button and visually check if the Diet interface is loaded. the tester will click the Exercise button and visually check if the Exercise interface is loaded. The tester will click the Rest button and visually check if the Rest interface is loaded

\end{enumerate}

\subsubsection{Diet Section Testing}

Testing all functional requirements for rest section of REVITALIZE. (Refer to BE4 in SRS)

\begin{enumerate}

\item{FR-DS-1\\}

Control: Manual
					
Initial State: Diet section is initialized for the first time and an initial information dialog is launched
					
Input: An event that loads the diet section for the first time
					
Output: A fillable dialog box is launched with height, dietary information, weight and calorie information

Test Case Derivation: Request is made to enter diet section for the first time

How test will be performed: Tester will enter rest section for the first time and visually check if the dialog box with correct input information is launched
					
\item{FR-DS-2\\}

Control: Manual
					
Initial State: Diet section is initialized for the first time and an initial information dialog is launched
					
Input: Initial information dialog values are filled
					
Output: Initial information values are saved to the database

Test Case Derivation: Initial information dialog is filled

How test will be performed: Tester will fill the initial information dialog after entering the diet section. The tester will visually check the user data in the database to see if the initial information is saved
					

\item{FR-DS-3\\}

Control: Manual
					
Initial State: section section is initialized with a list of food logged for the current calender day
					
Input: An event that loads the rest section
					
Output: A list of inputted food is loaded for the current calender day

Test Case Derivation: Request is made to enter rest section

How test will be performed: Tester will enter the rest section and will visually check if a list of logged food is loaded for the current calender day
					
\item{FR-DS-4 \\}

Control: Manual
					
Initial State: Diet section is displayed with add food button
					
Input: Click add food button
					
Output: A dialog box is launched that lets the user fill custom food information. the food is added to the food log list of the current calender day

Test Case Derivation: User clicks add food button

How test will be performed: Tester will click on add food button and will visuallly check if a fillable dialog box is launched. The tester will fill the dialog box information and click the add food button. The tester will visually check if the food has been added to the list of logged food for the current calender day

\item{FR-DS-5\\}

Control: Manual
					
Initial State: Diet section is displayed with search food button
					
Input: Click add search food button
					
Output: A recipe criteria user interface is launched that displays a list of modifiable criteria and a search button

Test Case Derivation: User clicks search food button

How test will be performed: Tester will click on search food exercise button and will visuallly check if a fillable dialog box is launched. The tester will fill the dialog box information and click the add food button. The tester will visually check if the food has been added to the list of logged food for the current calender day

\item{FR-WS-4\\}

Initial State: Each exercise in the workout section is displayed with an edit exercise button
					
Input: click edit exercise button
					
Output: A fillable dialog box is launched with information of the exercise. Once the edit exercise button is clicked, the dialog box will close and update the exercise information in the list of exercises for the current calender day

Test Case Derivation: User clicks edit exercise button

How test will be performed: Tester will click on edit exercise button and will change the infromation on the fillable dialog box. The tester will click the edit exercise button and will visually check if the information is changed on the exercise list for the current calender day.

\item{FR-WS-5\\}

Control: Manual
					
Initial State: Workout section is displayed with list of exercises for current calender day
					
Input: An event that loads the workout section
					
Output: If repitition and sets for exercises not logged, dialog box for exercise is launched and the missing repitition and set values are highlighted

Test Case Derivation: User launches the workout section. User adds an exercise and does not input repitition and set values.

How test will be performed: Tester will add exercise without inputting values for set and repititions. The user will visually check if the exercise dialog is launched with highlighted missing repititions and set values.

\end{enumerate}


\subsubsection{Workout Section Testing}

Testing all functional requirements for workout section of REVITALIZE. (Refer to BE5 in SRS)

\begin{enumerate}

\item{FR-WS-1\\}

Control: Manual
					
Initial State: Workout section is initialized with a preset list of exercises of the current calender day
					
Input: An event that loads the workout section
					
Output: A preset list of exercises is loaded for the current calender day

Test Case Derivation: Request is made to enter workout section

How test will be performed: Tester will enter the workout section and will visually check if a list of preset exercises are loaded for the current calender day
					
\item{FR-WS-2 \\}

Control: Manual
					
Initial State: Workout section is displayed with add exercise button
					
Input: Click add exercise button
					
Output: A dialog box is launched that lets the user fill custom exercise information. the exercise is added to the exercise list of the current calender day

Test Case Derivation: User clicks add exercise button

How test will be performed: Tester will click on add exercise button and will visuallly check if a fillable dialog box is launched. The tester will fill the dialog box information and click the add exercise button. The tester will visually check if the exercise has been added to the list of exercises for the current calender day

\item{FR-WS-3\\}

Initial State: Each exercise in the workout section is displayed with a delete exercise button
					
Input: Click delete exercise button
					
Output: The exercise is deleted from the exercise list of the current calender day

Test Case Derivation: User clicks delete exercise button

How test will be performed: Tester will click on delete exercise button and will visuallly check if the exercise is deleted from the list of exercises for the current calender day. 

\item{FR-WS-4\\}

Initial State: Each exercise in the workout section is displayed with an edit exercise button
					
Input: click edit exercise button
					
Output: A fillable dialog box is launched with information of the exercise. Once the edit exercise button is clicked, the dialog box will close and update the exercise information in the list of exercises for the current calender day

Test Case Derivation: User clicks edit exercise button

How test will be performed: Tester will click on edit exercise button and will change the infromation on the fillable dialog box. The tester will click the edit exercise button and will visually check if the information is changed on the exercise list for the current calender day.

\item{FR-WS-5\\}

Control: Manual
					
Initial State: Workout section is displayed with list of exercises for current calender day
					
Input: An event that loads the workout section
					
Output: If repitition and sets for exercises not logged, dialog box for exercise is launched and the missing repitition and set values are highlighted

Test Case Derivation: User launches the workout section. User adds an exercise and does not input repitition and set values.

How test will be performed: Tester will add exercise without inputting values for set and repititions. The user will visually check if the exercise dialog is launched with highlighted missing repititions and set values.

\end{enumerate}


\subsubsection{Rest Section Testing}

Testing all functional requirements for rest section of REVITALIZE. (Refer to BE6 in SRS)

\begin{enumerate}

\item{FR-RS-1\\}

Control: Manual
					
Initial State: Rest section is initialized with the sleep statistics of the current calender day
					
Input: An event that loads the rest section
					
Output: Sleep statistics are loaded for the current calender day

Test Case Derivation: Request is made to enter rest section

How test will be performed: Tester will enter the rest section and will visually check if slep statistics are loaded for the current calender day
					
\item{FR-RS-2 \\}

Control: Manual
					
Initial State: Rest section is initialized with the sleep statistics of the current calender day
					
Input: Alter sleep data
					
Output: The sleep data is updated with user changes

Test Case Derivation: User alters sleep data

How test will be performed: Tester will alter the sleep data and will visually check if the new values are correctly displayed in the sleep statistics

\end{enumerate}
...

\subsection{Tests for Nonfunctional Requirements}

\wss{The nonfunctional requirements for accuracy will likely just reference the
  appropriate functional tests from above.  The test cases should mention
  reporting the relative error for these tests.}

\wss{Tests related to usability could include conducting a usability test and
  survey.}

\subsubsection{Look and Feel Testing}

\begin{enumerate}

\item{LF1\\}

Type: Dynamic, Functional, Manual
					
Initial State: User is using REVITALIZE features
					
Input/Condition: REVITALIZE features are in use
					
Output/Result: All UI/UX design and elements matches original design and are displayed correctly and neatly
					
How test will be performed: All related stakeholders will test application with the focus on neatness and attractiveness of UI/UX design of REVITALIZE and answer question 1 of the Usability Survey. Would need an average rating of 8.5 or above out of 10 and assess all stakeholders responses to make improvements
					
\item{LF2\\}

Type: Static, Manual
					
Initial State: A display of all pages in REVITALIZE
					
Input/Condition: All displays for all pages in REVITALIZE are at a common point during a user session
					
Output/Result: All colours are considered appealing, contrasting and non-intrusive
					
How test will be performed: All related stakeholders will test application with the focus on colour and answer question 2 of the Usability Survey. Would need an average rating of 8.5 or above out of 10 for each factor and assess all stakeholders responses to make improvements

\end{enumerate}

\subsubsection{Usability and Humanity Testing}

\begin{enumerate}

\item{UH1\\}

Type: Dynamic, Functional, Manual
					
Initial State: User is using REVITALIZE features
					
Input/Condition: REVITALIZE features are in use, using one hand/one finger
					
Output/Result: REVITALIZE features are displaying correct outputs and results
					
How test will be performed: All related stakeholders with varying size hands/fingers can use all aspects of REVITALIZE using one hand/finger and have an average rating of 9.5 or above for question 3 of the Usability Survey
					
\item{UH2\\}

Type: Dynamic, Functional, Manual
					
Initial State: Main page of application is displayed
					
Input/Condition: User uses main page to access features (Diet, Workout and Rest Section)
					
Output/Result: All features take less than 10 seconds to access
					
How test will be performed: Stakeholders will navigate to the Diet, Workout and/or Rest section from the main page in 10 seconds or less. 90\% of stakeholders need to be able to navigate to any of the sections in 10 seconds or less

\item{UH3\\}

Type: Dynamic, Functional, Manual
					
Initial State: REVITALIZE is loaded but not in use
					
Input/Condition: Users in targeted demographic will use all features of REVITALIZE
					
Output/Result: Results gathered from survey
					
How test will be performed: Primary stakeholders such as teenagers and young adults 14 years or older will test application and fill in survey, with the goal of an approval rating of 85\% or above

\item{UH4\\}

Type: Dynamic, Functional, Manual
					
Initial State: REVITALIZE is loaded but not in use
					
Input/Condition: User will use all features of REVITALIZE
					
Output/Result: User can use and understand basic/common aspects of all features after 3rd iteration
					
How test will be performed: Stakeholders will use all features/aspects of REVITALIZE and 85\% of stakeholders should be able to use and understand basic/common aspects of all features in 3 iterations or less.

\item{UH5\\}

Type: Dynamic, Functional, Manual
					
Initial State: REVITALIZE is loaded but not in use
					
Input/Condition: User will use all features of REVITALIZE
					
Output/Result: Results gathered from survey based on consistency of UI aspects such as buttons, drop-downs, words etc.
					
How test will be performed: Stakeholders will test application with focus on consistency of UI aspects and fill in survey, with the goal of an approval rating of 85\% or above

\end{enumerate}

\subsubsection{Performance Testing}

\begin{enumerate}

\item{PE1\\}

Type: Dynamic, Functional, Manual
					
Initial State: REVITALIZE is loaded but not in use
					
Input/Condition: All REVITALIZE features are loaded with appropriate data
					
Output/Result: loading time of all REVITALIZE features
					
How test will be performed: Developers will run performance tests and ensure that all output data loads within 5 seconds or less for 95\% of all API responses and outputs
					
\item{PE2\\}

Type: Dynamic, Functional, Automatic
					
Initial State: REVITALIZE is loaded but not in use
					
Input/Condition: User uses all features of REVITALIZE where output contain data/numbers
					
Output/Result: data/numbers of used features in floating points
					
How test will be performed: Developers will run accuracy tests to ensure output data/numbers are accurate for double precision floating points and pass all test cases

\item{PE3\\}

Type: Dynamic, Functional, Manual
					
Initial State: REVITALIZE is loaded but not in use
					
Input/Condition: Multiple (More than 50) users using REVITALIZE
					
Output/Result: Performance metrics (ex. loading time, frames per second etc.)
					
How test will be performed: 50 or more users (does not have to be actual people) use/send requests to REVITALIZE application simultaneously and analyze performance trends based on the number of users

\item{PE4\\}

Type: Dynamic, Functional, Manual
					
Initial State: REVITALIZE is loaded but not in use
					
Input/Condition: User will use all features of REVITALIZE
					
Output/Result: Storage percentage of data
					
How test will be performed: Developers will try to add as much data as possible to user account and application should be able to store 1 million or more data points for all users

\end{enumerate}

\subsection{Traceability Between Test Cases and Requirements}

\wss{Provide a table that shows which test cases are supporting which
  requirements.}

\section{Unit Test Description}

\wss{Reference your MIS and explain your overall philosophy for test case
  selection.}  
\wss{This section should not be filled in until after the MIS has
  been completed.}

\subsection{Unit Testing Scope}

\wss{What modules are outside of the scope.  If there are modules that are
  developed by someone else, then you would say here if you aren't planning on
  verifying them.  There may also be modules that are part of your software, but
  have a lower priority for verification than others.  If this is the case,
  explain your rationale for the ranking of module importance.}

\subsection{Tests for Functional Requirements}

\wss{Most of the verification will be through automated unit testing.  If
  appropriate specific modules can be verified by a non-testing based
  technique.  That can also be documented in this section.}

\subsubsection{Module 1}

\wss{Include a blurb here to explain why the subsections below cover the module.
  References to the MIS would be good.  You will want tests from a black box
  perspective and from a white box perspective.  Explain to the reader how the
  tests were selected.}

\begin{enumerate}

\item{test-id1\\}

Type: \wss{Functional, Dynamic, Manual, Automatic, Static etc. Most will
  be automatic}
					
Initial State: 
					
Input: 
					
Output: \wss{The expected result for the given inputs}

Test Case Derivation: \wss{Justify the expected value given in the Output field}

How test will be performed: 
					
\item{test-id2\\}

Type: \wss{Functional, Dynamic, Manual, Automatic, Static etc. Most will
  be automatic}
					
Initial State: 
					
Input: 
					
Output: \wss{The expected result for the given inputs}

Test Case Derivation: \wss{Justify the expected value given in the Output field}

How test will be performed: 

\item{...\\}
    
\end{enumerate}

\subsubsection{Module 2}

...

\subsection{Tests for Nonfunctional Requirements}

\wss{If there is a module that needs to be independently assessed for
  performance, those test cases can go here.  In some projects, planning for
  nonfunctional tests of units will not be that relevant.}

\wss{These tests may involve collecting performance data from previously
  mentioned functional tests.}

\subsubsection{Module ?}
		
\begin{enumerate}

\item{test-id1\\}

Type: \wss{Functional, Dynamic, Manual, Automatic, Static etc. Most will
  be automatic}
					
Initial State: 
					
Input/Condition: 
					
Output/Result: 
					
How test will be performed: 
					
\item{test-id2\\}

Type: Functional, Dynamic, Manual, Static etc.
					
Initial State: 
					
Input: 
					
Output: 
					
How test will be performed: 

\end{enumerate}

\subsubsection{Module ?}

...

\subsection{Traceability Between Test Cases and Modules}

\wss{Provide evidence that all of the modules have been considered.}
				
\bibliographystyle{plainnat}

\bibliography{../../refs/References}

\newpage

\section{Appendix}

This is where you can place additional information.

\subsection{Symbolic Parameters}

The definition of the test cases will call for SYMBOLIC\_CONSTANTS.
Their values are defined in this section for easy maintenance.

\subsection{Usability Survey Questions?}

\wss{This is a section that would be appropriate for some projects.}

\begin{enumerate}
    \item How would you rate overall neatness of UI/UX design of REVITALIZE out of 10? What are elements you like related to neatness? What are elements you would improve related to neatness?
    \item Are colours used in REVITALIZE non-intrusive, appealing and/or contrasting? Rate each factor of non-intrusive, appealing and contrasting out of 10? What are elements you like that elevate these factors? What are elements you would improve to elevate these factors?
    \item How would you rate the overall ability to use REVITALIZE with only one hand/one finger out of 10? What are elements that help with this? What are elements you would improve?
\end{enumerate}

\end{document}