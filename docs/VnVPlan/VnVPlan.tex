\documentclass[12pt, titlepage]{article}
\usepackage[utf8]{inputenc}
\usepackage[margin=1in]{geometry}
\usepackage{booktabs}
\usepackage{tabularx}
\usepackage{hyperref}
\usepackage{float}
\hypersetup{
colorlinks,
citecolor=blue,
filecolor=black,
linkcolor=red,
urlcolor=blue
}
\usepackage[round]{natbib}
\usepackage{indentfirst}
\usepackage{enumerate}
\usepackage[shortlabels]{enumitem}
\usepackage{xcolor}
\usepackage{graphicx}
\usepackage{float}
\usepackage{multirow}

%% Comments

\usepackage{color}

\newif\ifcomments\commentstrue %displays comments
%\newif\ifcomments\commentsfalse %so that comments do not display

\ifcomments
\newcommand{\authornote}[3]{\textcolor{#1}{[#3 ---#2]}}
\newcommand{\todo}[1]{\textcolor{red}{[TODO: #1]}}
\else
\newcommand{\authornote}[3]{}
\newcommand{\todo}[1]{}
\fi

\newcommand{\wss}[1]{\authornote{blue}{SS}{#1}} 
\newcommand{\plt}[1]{\authornote{magenta}{TPLT}{#1}} %For explanation of the template
\newcommand{\an}[1]{\authornote{cyan}{Author}{#1}}

%% Common Parts

\newcommand{\progname}{REVITALIZE} % PUT YOUR PROGRAM NAME HERE
\newcommand{\authname}{Team 13, REVITALIZE
\\ Bill Nguyen
\\ Syed Bokhari
\\ Hasan Kibria
\\ Youssef Dahab
\\ Logan Brown
\\ Mahmoud Anklis} % AUTHOR NAMES                  

\usepackage{hyperref}
    \hypersetup{colorlinks=true, linkcolor=blue, citecolor=blue, filecolor=blue,
                urlcolor=blue, unicode=false}
    \urlstyle{same}
                                


\begin{document}

\title{Project Title: System Verification and Validation Plan for \progname{}} 
\author{\authname}
\date{\today}

\maketitle

\pagenumbering{roman}

\section{Revision History}

\begin{tabularx}{\textwidth}{p{3cm}p{2cm}X}
	\toprule {\bf Date} & {\bf Version} & {\bf Notes}\\
	\midrule
	October 31st, 2022 & Bill Nguyen & Added Functional Requirements Tests for Login Page\\
	October 31st, 2022 & Bill Nguyen & Added Non Functional Requirements Tests for Look and Feel, Usability and Performance\\
	October 31st, 2022 & Bill Nguyen & Added 3 Questions to Usability Survey\\
	November 1st, 2022 & Youssef Dahab & Added VnV Team, SRS Verification Plan\\
	November 1st, 2022 & Bill Nguyen & Self Reflection\\
	November 1st, 2022 & Youssef Dahab & Added Design Verification Plan\\
	November 1st, 2022 & Syed Bokhari & Added Functional Requirement Tests for Signup Page, Main Page, Diet Section, Workout Section, Rest Section\\
	November 1st, 2022 & Youssef Dahab & Added VnV Verification Plan\\
	November 1st, 2022 & Mahmoud Anklis & Added Non Functional Requirements Tests for Operational, Maintainability and Portability, Security and Cultural and Political\\
	November 1st, 2022 & Mahmoud Anklis & Added Self Reflection as well as Symbols, Abbreviations and Acronyms\\
	November 1st, 2022 & Youssef Dahab & Added Software Validation Plan\\
	November 1st, 2022 & Logan Brown & Added General Information and Self Reflection\\
	November 1st, 2022 & Logan Brown & Added Symbolic Parameters\\
	November 1st, 2022 & Youssef Dahab & Added Implementation Verification Plan\\
	\bottomrule
\end{tabularx}

\newpage

\tableofcontents

\listoftables


\listoffigures
\wss{Remove this section if it isn't needed}

\newpage

\section{Symbols, Abbreviations and Acronyms}

\renewcommand{\arraystretch}{1.2}
\begin{tabular}{l l} 
	\toprule		
	\textbf{symbol} & \textbf{description}\\
	\midrule 
	REVITALIZE & Name of application\\
	UAT & User Acceptance Testing\\
	UI/UX & User Interface/User Experience\\
	HCI & Human-Computer Interface\\
	MG & Module Guide\\
	MIS & Module Interface Specification\\
	SRS & Software Requirements Specification\\
	VnV & Verification and Validation\\
	FR & Functional Requirement\\
	NFR & Non Functional Requirement\\
	SE & Security Requirement\\
	LP & Login Page\\
	SP & Sign-up Page\\
	MP & Main Page or Maintainability and Portability Requirements\\
	DS & Diet Section\\
	WS & Workout Section\\
	RS & Rest Section\\
	LF & Look and Feel Requirements\\
	UH & Usability and Humanity Requirements\\
	PE & Performance Requirement\\
	OE & Operational Requirement\\
	SE & Security Requirement\\
	CU & Cultural Requirement\\
	\bottomrule
\end{tabular}\\

\wss{symbols, abbreviations or acronyms -- you can simply reference the SRS
	\citep{SRS} tables, if appropriate}

\newpage

\pagenumbering{arabic}

\noindent This document outlines the Verification \& Validation Plan of the REVITALIZE application.

\section{General Information}

\subsection{Summary}

\noindent This report describes the tests and validation that will be conducted on the REVITALIZE app. The REVITALIZE app is an all-in-one health and wellness app, comprised of 1 main section and 3 major subsections. The main section is a calendar which organizes and documents the contents of the 3 subsections. The 3 subsections are the diet section, workout section, and sleep section.

\subsection{Objectives}

\noindent This document outlines testing plans and testing descriptions in order to determine the following objectives: 
\begin{itemize}
	\item Demonstrate satisfactory usability based on laid out criteria
	\item Satisfy requirements outlined in SRS
	\item Build confidence that implementation behaves as intended
\end{itemize}


\subsection{Relevant Documentation}

\noindent This document references the SRS \citet{SRS}, the Module Guide \citet{MG}, and the Module Interface Specification \citet{MIS}


\section{Plan}

This section outlines REVITALIZE's VnV plan. It begins with identifying the VnV team. It then moves to discussing the SRS, Design, VnV, and implementation verification plans. This section then concludes with REVITALIZE's software verification plan.

\subsection{Verification and Validation Team}

\begin{table}[H]
	\centering
	\begin{tabular}{|p{3cm}|p{4cm}|p{7cm}|}
		\hline
		\multicolumn{1}{|c|}{\textbf{Name}} & \multicolumn{1}{|c|}{\textbf{Role}} & \multicolumn{1}{|c|}{\textbf{Description}}
		\\ \hline
		Hasan Kibria & Front-end Application Tester & Run tests to verify and validate front-end design of the application.
		\\ \hline
		Bill Nguyen & Back-end Application Tester & Run tests to verify and validate back-end design of the application.
		\\ \hline
		Syed Bokhari & Database Application Tester & Run tests to verify and validate database schema, tables, relationships, and data mappings. 
		\\ \hline
		Logan Brown & Performance Tester & Run tests to verify and validate application speed, responsiveness, and stability.
		\\ \hline
		Youssef Dahab & UI/UX Tester & Run tests to verify and validate application's accessibility, usability, efficiency, and safety to measure HCI aspects.
		\\ \hline  
		Mahmoud Anklis & System Integration Tester & Run tests to verify and validate front-end, back-end, and database components integration with each other.
		\\ \hline
		Primary \& Secondary Stakeholders & UAT & Perform user acceptance testing to verify and validate that the application meets its purpose.
		\\ \hline
		Dr. Smith & Course Instructor & Provide high level project feedback.
		\\ \hline
		Ting-Yu Wu & Course TA & Provide low level project feedback.
		\\ \hline
	\end{tabular}
\end{table}

REVITALIZE team members are aware that they might have to put on multiple hats or perform multiple roles.

\subsection{SRS Verification Plan}

The REVITALIZE team's SRS verification plan is to have the SRS reviewed by classmates, group members, and course TA. Classmates already adhocly reviewed REVITALIZE's SRS and provided their feedback. Furthermore, the course TA will review REVITALIZE's SRS to provide more detailed and structured feedback. REVITALIZE team members will meet together to gather all feedback from classmates, other team members and course TA. The team will then assess the feedback and analyze what can be incorporated. After that, the SRS will be refined and updated as part of the SRS verification process.

\subsection{Design Verification Plan}

The REVITALIZE team's design verification plan is to have its design reviewed by classmates, group members, and course TA. The team will gather all feedback from classmates, other group members, and the course TA and incorporate it in the design document. After completing the MG and MIS, the team will ensure that their modules match the system requirements in the SRS. Finally, a design walk-through will be conducted by the team to verify the design.

\subsection{Verification and Validation Plan Verification Plan}

The REVITALIZE teams plan to verify its VnV plan is to have it reviewed by classmates, group members, and course TA. The team will gather all feedback and incorporate it in the VnV plan. The REVITALIZE team will also conduct a walk-through after feedback has been incorporated to verify that every functional and nonfunctional requirement in the SRS has at least one test that tests it in the VnV plan. Team members will then run the tests in the VnV plan to determine if the system requirements stated in the SRS have been met. This way, the team verifies the completeness of the VnV in both directions.\\
The same process will be followed regarding the unit tests. Each module in the MG and MIS will be matched to the tests written in the VnV plan to verify it.

\subsection{Implementation Verification Plan}

REVITALIZE team members will use both a static and dynamic approach to verify the application's implementation. In terms of dynamic methods, automated unit tests will be performed as part of testing the system tests mentioned in section 6. Furthermore, test cases will be monitored to ensure its up to date such that full testing coverage of the code can be done. This includes edge case testing.\\\\
A prototype testing method will also be used. REVITALIZE will not be implemented in one go. Rather, features and functionality will constantly be added throughout the course of the project. Therefore, REVITALIZE will go through numerous prototypes or phases. This prototype testing will act as an ongoing static and dynamic verification of the REVITALIZE application.\\\\
Static verification of REVITALIZE's implementation will be done by performing code inspections. Team members will perform code reviews as part of static analysis where each team member will follow the guidelines stated in REVITALIZE's Development Plan. Team members will review each other's code and raise issues to note any concerns. The purpose of conducting static verification of the implementation in this way is to ensure code walk-throughs are completed, concerns are raised, and best implementation practices are followed.

\subsection{Automated Testing and Verification Tools}

Automated testing and verification tools have been previously stated in sections 6 and 7 of REVITALIZE's \href{https://github.com/BillNguyen1999/REVITALIZE/blob/main/docs/DevelopmentPlan/DevelopmentPlan.pdf}{\color{blue}Development Plan}.

\subsection{Software Validation Plan}

The REVITALIZE team's software validation plan includes having a review session with primary and secondary stakeholders to check that the SRS requirements document captures the right requirements. After REVITALIZE's initial release, another session with the stakeholders will be held to validate the implementation. The user will use REVITALIZE for the first time, answer team member questions, and give feedback on the application. After that, as per the feedback, the SRS and the implementation may require modifications such as the addition or removal of particular functionality.

\section{System Test Description}

\subsection{Tests for Functional Requirements}

Subsections of the requirements will be divided into the events from our SRS, which are Login Page, Sign up Page, Main Page, Diet Section, Workout Section and Rest Section. There will be 1 test per functional requirement, and will follow the same order as functional requirements in SRS (ex. FR1 in VnV plan is the test for FR1 in SRS).

\subsubsection{Login Page Testing}

Testing all functional requirements for login page of REVITALIZE. (Refer to BE1 in SRS)

\begin{enumerate}
	
	\item{FR-LP-1\\}
	
	Control: Manual
	
	Initial State: Loading stage of the login page
	
	Input: An event that loads the login page
	
	Output: Login page is displayed with all necessary components
	
	Test Case Derivation: Request is made to load login page
	
	How test will be performed: Tester will open REVITALIZE application and login page should be displayed
	
	\item{FR-LP-2\\}
	
	Control: Manual
	
	Initial State: Login page is displayed with username textbox
	
	Input: Enter username information in textbox
	
	Output: Username information entered is displayed in textbox
	
	Test Case Derivation: User can enter information in username textbox
	
	How test will be performed: Tester will enter information in username textbox and checks if textbox displays what the tester entered.
	
	\item{FR-LP-3\\}
	
	Control: Manual
	
	Initial State: Login page is displayed with password textbox
	
	Input: Enter password information in textbox
	
	Output: password information entered is displayed in textbox via hidden text
	
	Test Case Derivation: User can enter information in password textbox
	
	How test will be performed: Tester will enter information in password textbox and checks if textbox displays what the tester entered via hidden text.
	
	\item{FR-LP-4\\}
	
	Control: Manual
	
	Initial State: Login page is displayed with login button
	
	Input: Click login button
	
	Output: Intended events occurs. Refer to FR9
	
	Test Case Derivation: User clicks login button and a request is made based on username and password text-boxes
	
	How test will be performed: Tester will click on login button and check if request is made correctly
	
	\item{FR-LP-5\\}
	
	Control: Manual
	
	Initial State: Login page is displayed with forgot password button
	
	Input: Click forgot password button
	
	Output: Display forgot password screen with textbox to enter email
	
	Test Case Derivation: User clicks forgot password button and request is made to display forgot password screen with textbox to enter email
	
	How test will be performed: Tester will click on forgot password button and checks if forgot password screen is displayed with textbox to enter email
	
	\item{FR-LP-6\\}
	
	Control: Manual
	
	Initial State: Login page is displayed with stay logged in checkbox that is empty
	
	Input: Click stay logged in checkbox
	
	Output: Display a check-mark in the stay logged in checkbox if checkbox is empty, else if checkbox contains check-mark already it will then display an empty checkbox
	
	Test Case Derivation: User clicks on stay logged in checkbox and displays appropriate action
	
	How test will be performed: Tester will click on checkbox and checks to see if check-mark is displayed if checkbox was empty and if an empty checkbox appears if checkbox contained a check-mark
	
	\item{FR-LP-7\\}
	
	Control: Manual
	
	Initial State: Loading stage of REVITALIZE where previous state had stay logged in checkbox checked
	
	Input: An event that loads REVITALIZE
	
	Output: Display main page, with same data from previous state of main page
	
	Test Case Derivation: User can load REVITALIZE and main page is displayed with same data as the previous time user opened REVITALIZE main page
	
	How test will be performed: Tester will check stay logged in checkbox go to main page, leave REVITALIZE, reopen REVITALIZE and check whether same data from main page is the same from the last time tester opened main page
	
	\item{FR-LP-8\\}
	
	Control: Manual
	
	Initial State: Login page is displayed with sign up button
	
	Input: Click sign up button
	
	Output: Loads and displays sign up page
	
	Test Case Derivation: User can click sign up button which loads and displays sign up page
	
	How test will be performed: Tester will click on sign up button and checks if sign up page is displayed
	
	\item{FR-LP-9\\}
	
	Control: Manual
	
	Initial State: Login page is displayed with inputted information in username and password text-boxes
	
	Input: Click login button
	
	Output: if failure state, display an invalid password or username banner, else if success state, load and display main page
	
	Test Case Derivation: User clicks login button and request is made based on username and password, and will proceed to main page only if username and password are valid
	
	How test will be performed: Tester will click on login button and test for scenarios when login should be successful and when login should fail
	
	
\end{enumerate}

\subsubsection{Signup Page Testing}

Testing all functional requirements for sign page of REVITALIZE. (Refer to BE2 in SRS)

\begin{enumerate}
	
	\item{FR-SP-1\\}
	
	Control: Manual
	
	Initial State: Signup page is displayed with username textbox
	
	Input: Enter username information in textbox
	
	Output: Username information entered is displayed in textbox
	
	Test Case Derivation: User can enter information in username textbox
	
	How test will be performed: Tester will enter information in username textbox and checks if textbox displays what the tester entered
	
	\item{FR-SP-2 \\}
	
	Control: Manual
	
	Initial State: Signup page is displayed with password textbox
	
	Input: Enter password information in textbox
	
	Output: Password information entered is displayed in textbox via hidden text
	
	Test Case Derivation: User can enter information in password textbox
	
	How test will be performed: Tester will enter information in password textbox and checks if textbox displays what the tester entered via hidden text
	
	\item{FR-SP-3\\}
	
	Control: Manual
	
	Initial State: signup page is displayed with email textbox
	
	Input: Enter email information in textbox
	
	Output: Email information entered is displayed in textbox
	
	Test Case Derivation: User can enter information in email textbox
	
	How test will be performed: Tester will enter information in email textbox and checks if textbox displays what the tester entered
	
	\item{FR-SP-4\\}
	
	Control: Manual
	
	Initial State: Signup page is displayed with signup button
	
	Input: Click signup button
	
	Output: Intended events occurs. Refer to FR14
	
	Test Case Derivation: User clicks signup button and a request is made based on username, password and email text-boxes
	
	How test will be performed: Tester will click on signup button and check if request is made correctly
	
	\item{FR-SP-5\\}
	
	Control: Manual
	
	Initial State: Signup page is displayed with inputted information in username and password text-boxes
	
	Input: Click signup button
	
	Output: if failure state, display an invalid username, password or email banner, else if success state, load and display login page
	
	Test Case Derivation: User clicks signup button and request is made based on username, password and email, and will proceed to login page only if username, password and email are valid
	
	How test will be performed: Tester will click on signup button and test for scenarios when signup should be successful and when signup should fail
	
	
\end{enumerate}

\subsubsection{Main Page Testing}

Testing all functional requirements for main page of REVITALIZE. (Refer to BE3 in SRS)

\begin{enumerate}
	
	\item{FR-MP-1\\}
	
	Control: Manual
	
	Initial State: Main page is displayed with calender of current date
	
	Input: An event that loads the main page
	
	Output: Main page is displayed with all necessary components
	
	Test Case Derivation: Request is made to load main page
	
	How test will be performed: Tester will successfully login to the REVITALIZE application and will visually check if the correct calender data is loaded
	
	\item{FR-MP-2\\}
	
	Control: Manual
	
	Initial State: Main page and Diet, Workout, Rest sections are displayed with previous day and next day buttons
	
	Input: An event that loads the main page, Diet, Workout, Rest sections and the previous day and next day  buttons are clicked
	
	Output: Main page, Diet, Workout, Rest sections are displayed with previous day and next day buttons. Once the next day button is clicked, the calender refreshes  the calender information for the next day. Once the preivous day button is clicked, the calender refreshes the calender information for the previous day.
	
	Test Case Derivation: Request is made to load main page, Diet, Workout, Rest sections and the previous day and next day buttons are clicked
	
	How test will be performed: Tester will successfully login to the REVITALIZE application and will visually check if the previous day and next day buttons are visible. The tester will click the previous day button and visually check if the calender is updated to the date of the previous day. The tester will click the next day button and visually check if the calender is updated to the date of the next day. The tester will enter the Diet, Workout and Rest sections and will visually check if the previous day and next day buttons are visible. The tester will click the previous day button and visually check if the calender is updated to the date of the previous day. The tester will click the next day button and visually check if the calender is updated to the date of the next day
	
	\item{FR-MP-3\\}
	
	Control: Manual
	
	Initial State: Each interaction after leaving the main page must have a visible back button
	
	Input: An event that loads the next user interface after leaving the main page and the back button is clicked
	
	Output: The next user interface after leaving the main page iis displayed with a back button. Once the back button is clicked, the main page is loaded
	
	Test Case Derivation: Request is to leave the main page and the back button is clicked
	
	How test will be performed: Tester will leave the main page by selecting any of the options on the page. The tester will visually check if a back button is visible on every page that is entered through the main page interaction. The tester will click the back button and visually check if the current page is closed and the main page is loaded. The tester will repeat this process with every page that is  loaded from clicking an interaction from the main page
	
	\item{FR-MP-4\\}
	
	Control: Manual
	
	Initial State: Main page is displayed with Diet, Exercise and Rest buttons available to click
	
	Input: An event that loads the main page and the Diet, Exercise and Rest buttons are clicked
	
	Output: Main page is displayed with Diet, Exercise and Rest buttons. If the Diet button is clicked, the Diet interface is loaded. If the Exercise button is clicked, the Exercise interface is loaded. If the Rest button is clicked, the Rest interface is loaded
	
	Test Case Derivation: Request is made to load main page and the Diet, Exercise and Rest buttons are clicked
	
	How test will be performed: Tester will successfully login to the REVITALIZE application and will visually check if the Diet, Exercise and Rest buttons are visible. The tester will click the Diet button and visually check if the Diet interface is loaded. the tester will click the Exercise button and visually check if the Exercise interface is loaded. The tester will click the Rest button and visually check if the Rest interface is loaded
	
\end{enumerate}

\subsubsection{Diet Section Testing}

Testing all functional requirements for rest section of REVITALIZE. (Refer to BE4 in SRS)

\begin{enumerate}
	
	\item{FR-DS-1\\}
	
	Control: Manual
	
	Initial State: Diet section is initialized for the first time and an initial information dialog is launched
	
	Input: An event that loads the diet section for the first time
	
	Output: A fillable dialog box is launched with height, dietary information, weight and calorie information
	
	Test Case Derivation: Request is made to enter diet section for the first time
	
	How test will be performed: Tester will enter rest section for the first time and visually check if the dialog box with correct input information is launched
	
	\item{FR-DS-2\\}
	
	Control: Manual
	
	Initial State: Diet section is initialized for the first time and an initial information dialog is launched
	
	Input: Initial information dialog values are filled
	
	Output: Initial information values are saved to the database
	
	Test Case Derivation: Initial information dialog is filled
	
	How test will be performed: Tester will fill the initial information dialog after entering the diet section. The tester will visually check the user data in the database to see if the initial information is saved
	
	
	\item{FR-DS-3\\}
	
	Control: Manual
	
	Initial State: section section is initialized with a list of food logged for the current calender day
	
	Input: An event that loads the rest section
	
	Output: A list of inputted food is loaded for the current calender day
	
	Test Case Derivation: Request is made to enter rest section
	
	How test will be performed: Tester will enter the rest section and will visually check if a list of logged food is loaded for the current calender day
	
	\item{FR-DS-4 \\}
	
	Control: Manual
	
	Initial State: Diet section is displayed with add food button
	
	Input: Click add food button
	
	Output: A user interface is launched that lets the user select between searching for food or adding a custom meal
	
	Test Case Derivation: User clicks add food button
	
	How test will be performed: Tester will click on add food button and will visuallly check if the user interface with options for adding a custom meal and searching for food are available.
	
	\item{FR-DS-5\\}
	
	Control: Manual
	
	Initial State: Food adding user interface is displayed with search food button
	
	Input: Click search food button
	
	Output: A recipe criteria user interface is launched that displays a list of modifiable criteria and a search button
	
	Test Case Derivation: User clicks search food button
	
	How test will be performed: Tester will click on search food exercise button and will visuallly check if a recipe criteria user interface is launched that displays a list of modifiable criteria and a search button
	
	\item{FR-DS-6\\}
	
	Control: Manual
	
	Initial State: Recipe criteria user interface is launched
	
	Input: Search criteria is modified and search button is clicked
	
	Output: List of recipes are loaded correctly based on constraints of search criteria
	
	Test Case Derivation: User modifies the search criteria and clicks the search button
	
	How test will be performed: Tester will modify the search criteria and click the search button. The tester will visually check if a recipe list is loaded. The tester will visually check the correctness of the list based on selected constraints.
	
	\item{FR-DS-7\\}
	
	Control: Manual
	
	Initial State: Recipe list is loaded based on search constraints
	
	Input: Add recipe button is clicked
	
	Output: Selected recipe is added to the list of food logged on the current calendar day
	
	Test Case Derivation: User selects the recipe from the recipe list and clicks the add recipe button
	
	How test will be performed: Tester will select a recipe and click the add recipe button. The tester will visually check if the selected recipe has been added to the list of food logged on the current calendar day
	
	\item{FR-DS-8 \\}
	
	Control: Manual
	
	Initial State: Food adding interface is displayed with add custom meal button
	
	Input: Click add custom meal button
	
	Output: A dialog box is launched that lets the user fill custom meal information. The meal is added to the food log list of the current calendar day
	
	Test Case Derivation: User clicks add custom meal button
	
	How test will be performed: Tester will click on add custom meal button and will visually check if a fillable dialog box is launched. The tester will fill the dialog box information and click the add custom meal button. The tester will visually check if the meal has been added to the list of logged food for the current calendar day
	
\end{enumerate}


\subsubsection{Workout Section Testing}

Testing all functional requirements for workout section of REVITALIZE. (Refer to BE5 in SRS)

\begin{enumerate}
	
	\item{FR-WS-1\\}
	
	Control: Manual
	
	Initial State: Workout section is initialized with a preset list of exercises of the current calendar day
	
	Input: An event that loads the workout section
	
	Output: A preset list of exercises is loaded for the current calendar day
	
	Test Case Derivation: Request is made to enter workout section
	
	How test will be performed: Tester will enter the workout section and will visually check if a list of preset exercises are loaded for the current calendar day
	
	\item{FR-WS-2 \\}
	
	Control: Manual
	
	Initial State: Workout section is displayed with add exercise button
	
	Input: Click add exercise button
	
	Output: A dialog box is launched that lets the user fill custom exercise information. The exercise is added to the exercise list of the current calendar day
	
	Test Case Derivation: User clicks add exercise button
	
	How test will be performed: Tester will click on add exercise button and will visually check if a fillable dialog box is launched. The tester will fill the dialog box information and click the add exercise button. The tester will visually check if the exercise has been added to the list of exercises for the current calendar day
	
	\item{FR-WS-3\\}
	
	Initial State: Each exercise in the workout section is displayed with a delete exercise button
	
	Input: Click delete exercise button
	
	Output: The exercise is deleted from the exercise list of the current calendar day
	
	Test Case Derivation: User clicks delete exercise button
	
	How test will be performed: Tester will click on delete exercise button and will visually check if the exercise is deleted from the list of exercises for the current calendar day. 
	
	\item{FR-WS-4\\}
	
	Initial State: Each exercise in the workout section is displayed with an edit exercise button
	
	Input: click edit exercise button
	
	Output: A fillable dialog box is launched with information of the exercise. Once the edit exercise button is clicked, the dialog box will close and update the exercise information in the list of exercises for the current calendar day
	
	Test Case Derivation: User clicks edit exercise button
	
	How test will be performed: Tester will click on edit exercise button and will change the information on the fillable dialog box. The tester will click the edit exercise button and will visually check if the information is changed on the exercise list for the current calendar day.
	
	\item{FR-WS-5\\}
	
	Control: Manual
	
	Initial State: Workout section is displayed with list of exercises for current calendar day
	
	Input: An event that loads the workout section
	
	Output: If repetition and sets for exercises not logged, dialog box for exercise is launched and the missing repetition and set values are highlighted
	
	Test Case Derivation: User launches the workout section. User adds an exercise and does not input repetition and set values.
	
	How test will be performed: Tester will add exercise without inputting values for set and repetitions. The user will visually check if the exercise dialog is launched with highlighted missing repetitions and set values.
	
\end{enumerate}


\subsubsection{Rest Section Testing}

Testing all functional requirements for rest section of REVITALIZE. (Refer to BE6 in SRS)

\begin{enumerate}
	
	\item{FR-RS-1\\}
	
	Control: Manual
	
	Initial State: Rest section is initialized with the sleep statistics of the current calendar day
	
	Input: An event that loads the rest section
	
	Output: Sleep statistics are loaded for the current calendar day
	
	Test Case Derivation: Request is made to enter rest section
	
	How test will be performed: Tester will enter the rest section and will visually check if sleep statistics are loaded for the current calendar day
	
	\item{FR-RS-2 \\}
	
	Control: Manual
	
	Initial State: Rest section is initialized with the sleep statistics of the current calendar day
	
	Input: Alter sleep data
	
	Output: The sleep data is updated with user changes
	
	Test Case Derivation: User alters sleep data
	
	How test will be performed: Tester will alter the sleep data and will visually check if the new values are correctly displayed in the sleep statistics
	
\end{enumerate}
...

\subsection{Tests for Nonfunctional Requirements}

\subsubsection{Look and Feel Testing}

\begin{enumerate}
	
	\item{NFR-LF1\\}
	
	Type: Dynamic, Functional, Manual
	
	Initial State: User is using REVITALIZE features
	
	Input/Condition: REVITALIZE features are in use
	
	Output/Result: All UI/UX design and elements matches original design and are displayed correctly and neatly
	
	How test will be performed: All related stakeholders will test application with the focus on neatness and attractiveness of UI/UX design of REVITALIZE and answer question 1 of the Usability Survey. Would need an average rating of \hyperlink{MINIMUM_TEST_SCORE}{MINIMUM\_TEST\_SCORE} or above out of 10 and assess all stakeholders responses to make improvements
	
	\item{NFR-LF2\\}
	
	Type: Static, Manual
	
	Initial State: A display of all pages in REVITALIZE
	
	Input/Condition: All displays for all pages in REVITALIZE are at a common point during a user session
	
	Output/Result: All colours are considered appealing, contrasting and non-intrusive
	
	How test will be performed: All related stakeholders will test application with the focus on colour and answer question 2 of the Usability Survey. Would need an average rating of \hyperlink{MINIMUM_TEST_SCORE}{MINIMUM\_TEST\_SCORE} or above out of 10 for each factor and assess all stakeholders responses to make improvements
	
\end{enumerate}

\subsubsection{Usability and Humanity Testing}

\begin{enumerate}
	
	\item{NFR-UH1\\}
	
	Type: Dynamic, Functional, Manual
	
	Initial State: User is using REVITALIZE features
	
	Input/Condition: REVITALIZE features are in use, using one hand/one finger
	
	Output/Result: REVITALIZE features are displaying correct outputs and results
	
	How test will be performed: All related stakeholders with varying size hands/fingers can use all aspects of REVITALIZE using one hand/finger and have an average rating of \hyperlink{MINIMUM_TEST_SCORE_2}{MINIMUM\_TEST\_SCORE\_2} or above for question 3 of the Usability Survey
	
	\item{NFR-UH2\\}
	
	Type: Dynamic, Functional, Manual
	
	Initial State: Main page of application is displayed
	
	Input/Condition: User uses main page to access features (Diet, Workout and Rest Section)
	
	Output/Result: All features take less than \hyperlink{MINIMUM_ACCESS_TIME}{MINIMUM\_ACCESS\_TIME} seconds to access
	
	How test will be performed: Stakeholders will navigate to the Diet, Workout and/or Rest section from the main page in \hyperlink{MAXIMUM_ACCESS_TIME}{MAXIMUM\_ACCESS\_TIME} seconds or less. 90\% of stakeholders need to be able to navigate to any of the sections in \hyperlink{MAXIMUM_ACCESS_TIME}{MAXIMUM\_ACCESS\_TIME} seconds or less
	
	\item{NFR-UH3\\}
	
	Type: Dynamic, Functional, Manual
	
	Initial State: REVITALIZE is loaded but not in use
	
	Input/Condition: Users in targeted demographic will use all features of REVITALIZE
	
	Output/Result: Results gathered from survey
	
	How test will be performed: Primary stakeholders such as teenagers and young adults 14 years or older will test application and fill in survey, with the goal of an approval rating of \hyperlink{MIN_APPROVAL_RATING}{MIN\_APPROVAL\_RATING} or above
	
	\item{NFR-UH4\\}
	
	Type: Dynamic, Functional, Manual
	
	Initial State: REVITALIZE is loaded but not in use
	
	Input/Condition: User will use all features of REVITALIZE
	
	Output/Result: User can use and understand basic/common aspects of all features after 3rd iteration
	
	How test will be performed: Stakeholders will use all features/aspects of REVITALIZE and \hyperlink{MIN_APPROVAL_RATING}{MIN\_APPROVAL\_RATING} of stakeholders should be able to use and understand basic/common aspects of all features in 3 iterations or less.
	
	\item{NFR-UH5\\}
	
	Type: Dynamic, Functional, Manual
	
	Initial State: REVITALIZE is loaded but not in use
	
	Input/Condition: User will use all features of REVITALIZE
	
	Output/Result: Results gathered from survey based on consistency of UI aspects such as buttons, drop-downs, words etc.
	
	How test will be performed: Stakeholders will test application with focus on consistency of UI aspects and fill in survey, with the goal of an approval rating of \hyperlink{MIN_APPROVAL_RATING}{MIN\_APPROVAL\_RATING} or above
	
\end{enumerate}

\subsubsection{Performance Testing}

\begin{enumerate}
	
	\item{NFR-PE1\\}
	
	Type: Dynamic, Functional, Manual
	
	Initial State: REVITALIZE is loaded but not in use
	
	Input/Condition: All REVITALIZE features are loaded with appropriate data
	
	Output/Result: loading time of all REVITALIZE features
	
	How test will be performed: Developers will run performance tests and ensure that all output data loads within 5 seconds or less for \hyperlink{MIN_APPROVAL_RATING_2}{MIN\_APPROVAL\_RATING\_2} of all API responses and outputs
	
	\item{NFR-PE2\\}
	
	Type: Dynamic, Functional, Automatic
	
	Initial State: REVITALIZE is loaded but not in use
	
	Input/Condition: User uses all features of REVITALIZE where output contain data/numbers
	
	Output/Result: data/numbers of used features in floating points
	
	How test will be performed: Developers will run accuracy tests to ensure output data/numbers are accurate for double precision floating points and pass all test cases
	
	\item{NFR-PE3\\}
	
	Type: Dynamic, Functional, Manual
	
	Initial State: REVITALIZE is loaded but not in use
	
	Input/Condition: Multiple (More than \hyperlink{MIN_USER_LOAD}{MIN\_USER\_LOAD}) users using REVITALIZE
	
	Output/Result: Performance metrics (ex. loading time, frames per second etc.)
	
	How test will be performed: \hyperlink{MIN_USER_LOAD}{MIN\_USER\_LOAD} or more users (does not have to be actual people) use/send requests to REVITALIZE application simultaneously and analyze performance trends based on the number of users
	
	\item{NFR-PE4\\}
	
	Type: Dynamic, Functional, Manual
	
	Initial State: REVITALIZE is loaded but not in use
	
	Input/Condition: User will use all features of REVITALIZE
	
	Output/Result: Storage percentage of data
	
	How test will be performed: Developers will try to add as much data as possible to user account and application should be able to store \hyperlink{MIN_DATA_POINTS}{MIN\_DATA\_POINTS} or more data points for all users
	
\end{enumerate}

\subsubsection{Operational Testing}

\begin{enumerate}
	
	\item{NFR-OE1\\}
	
	Type: Dynamic, Functional, Manual
	
	Initial State: Internet connection is established on the device
	
	Input/Condition: REVITALIZE application is launched
	
	Output/Result: REVITALIZE features are all loaded 
	
	How test will be performed: Users will connect to the internet on their devices and launch the application. The application should load all features. 
	
	
\end{enumerate}

\subsubsection{Maintainability and Portability Testing}

\begin{enumerate}
	
	\item{NFR-MP1\\}
	
	Type: Static, Manual
	
	Initial State: Existing source code
	
	Input/Condition: New source code is added
	
	Output/Result: New source code is commented  
	
	How test will be performed: Developers will cross-check each other's comments each time new source code is pushed and before it is merged to the main repository 
	
	\item{NFR-MP2\\}
	
	Type: Static, Automatic
	
	Initial State: Existing source code
	
	Input/Condition: New source code is added
	
	Output/Result: New source code that is added adheres to Google JS Style Guide (refer to SRS)
	
	How test will be performed: The ESLint linter tool will scan the source code and ensure it adheres to the Google JS Style Guide for newly developed source code.  
	
\end{enumerate}

\subsubsection{Security Testing}

\begin{enumerate}
	
	\item{NFR-SE1\\}
	
	Type: Dynamic, Functional, Automatic
	
	Initial State: REVITALIZE is launched
	
	Input/Condition: User creates a new account
	
	Output/Result: REVITALIZE sign-up/account features are loaded and secured  
	
	How test will be performed: Developers will execute security tests to ensure account information is private and secure. 100\% of the test cases must pass
	
	\item{NFR-SE2\\}
	
	Type: Dynamic, Functional, Automatic
	
	Initial State: REVITALIZE sign-up feature is loaded 
	
	Input/Condition: User inputs an email already associated with an existing account
	
	Output/Result: Message alerts the user that user already exists with this email.  
	
	How test will be performed: Developers will run tests on the Database that stores the users that check to make sure there is no duplication. 100\% of the tests must pass  
	
\end{enumerate}

\subsubsection{Cultural and Political Testing}

\begin{enumerate}
	
	\item{NFR-CU1\\}
	
	Type: Dynamic, Functional, Manual
	
	Initial State: REVITALIZE is launched
	
	Input/Condition: REVITALIZE features are all loaded
	
	Output/Result: English is displayed in all REVITALIZE features
	
	How test will be performed: Developers will test the Front-End to ensure English is the language displayed. 
	
	
\end{enumerate}

\newpage
\subsection{Traceability Between Test Cases and Requirements}

\begin{table}[H]
	\begin{center}
		\caption{\textbf{Traceability Matrix for Login Page Functional Requirements}}
		\begin{tabularx}{\textwidth}{cc|c|c|c|c|c|c|c|c|c|c|c|c|}
			\cline{3-11}
			& & \multicolumn{9}{ c|}{Requirements} \\ \cline{3-11}
			& & FR1  & FR2 & FR3 & FR4 & FR5 & FR6 & FR7 & FR8 & FR9 \\ \cline{1-11}
			\multicolumn{1}{ |c| }{\multirow{10}{*}{Test Cases} } &
			\multicolumn{1}{|c| }{ FR-LP-1} &X&&&&&&&& \\ \cline{2-11}
			\multicolumn{1}{|c| }{} 	                  &
			\multicolumn{1}{|c| }{ FR-LP-2} &&X&&&&&&& \\ \cline{2-11}
			\multicolumn{1}{|c| }{} 	                  &
			\multicolumn{1}{|c| }{ FR-LP-3} &&&X&&&&&& \\ \cline{2-11}
			\multicolumn{1}{|c| }{} 	                  &
			\multicolumn{1}{|c| }{ FR-LP-4} &&&&X&&&&& \\ \cline{2-11}
			\multicolumn{1}{|c| }{}                        &
			\multicolumn{1}{|c| }{ FR-LP-5} &&&&&X&&&& \\ \cline{2-11}
			\multicolumn{1}{|c| }{} 	                  &
			\multicolumn{1}{|c| }{ FR-LP-6} &&&&&&X&&& \\ \cline{2-11}
			\multicolumn{1}{|c| }{} 	                  &
			\multicolumn{1}{|c| }{ FR-LP-7} &&&&&&&X&& \\ \cline{2-11}
			\multicolumn{1}{|c| }{}                        &
			\multicolumn{1}{|c| }{ FR-LP-8} &&&&&&&&X& \\ \cline{2-11}
			\multicolumn{1}{|c| }{}                        &
			\multicolumn{1}{|c| }{ FR-LP-9} &&&&&&&&&X \\ \cline{1-11}
		\end{tabularx}
	\end{center}
\end{table}

\begin{table}[H]
	\begin{center}
		\caption{\textbf{Traceability Matrix for Signup Page Functional Requirements}}
		\begin{tabularx}{\textwidth}{cc|c|c|c|c|c|c|c|c|}
			\cline{3-7}
			& & \multicolumn{5}{ c|}{Requirements} \\ \cline{3-7}
			& & FR10  & FR11 & FR12 & FR13 & FR14 \\ \cline{1-7}
			\multicolumn{1}{ |c| }{\multirow{6}{*}{Test Cases} } &
			\multicolumn{1}{|c| }{ FR-SP-1} &X&&&& \\ \cline{2-7}
			\multicolumn{1}{|c| }{} 	                  &
			\multicolumn{1}{|c| }{ FR-SP-2} &&X&&& \\ \cline{2-7}
			\multicolumn{1}{|c| }{} 	                  &
			\multicolumn{1}{|c| }{ FR-SP-3} &&&X&& \\ \cline{2-7}
			\multicolumn{1}{|c| }{} 	                  &
			\multicolumn{1}{|c| }{ FR-SP-4} &&&&X& \\ \cline{2-7}
			\multicolumn{1}{|c| }{}                        &
			\multicolumn{1}{|c| }{ FR-SP-5} &&&&&X \\ \cline{1-7}
		\end{tabularx}
	\end{center}
\end{table}

\begin{table}[H]
	\begin{center}
		\caption{\textbf{Traceability Matrix for Main Page Functional Requirements}}
		\begin{tabularx}{\textwidth}{cc|c|c|c|c|c|c|c|c|}
			\cline{3-7}
			& & \multicolumn{4}{ c|}{Requirements} \\ \cline{3-7}
			& & FR15  & FR16 & FR17 & FR18 & FR30 \\ \cline{1-7}
			\multicolumn{1}{ |c| }{\multirow{5}{*}{Test Cases} } &
			\multicolumn{1}{|c| }{ FR-MP-1} &X&&&& \\ \cline{2-7}
			\multicolumn{1}{|c| }{} 	                  &
			\multicolumn{1}{|c| }{ FR-MP-2} &&X&&&X \\ \cline{2-7}
			\multicolumn{1}{|c| }{} 	                  &
			\multicolumn{1}{|c| }{ FR-MP-3} &&&X&& \\ \cline{2-7}
			\multicolumn{1}{|c| }{} 	                  &
			\multicolumn{1}{|c| }{ FR-MP-4} &&&&X& \\ \cline{1-7}
		\end{tabularx}
	\end{center}
\end{table}

\begin{table}[H]
	\begin{center}
		\caption{\textbf{Traceability Matrix for Diet Page Functional Requirements}}
		\begin{tabularx}{\textwidth}{cc|c|c|c|c|c|c|c|c|c|c|c|c|c|}
			\cline{3-12}
			& & \multicolumn{10}{ c|}{Requirements} \\ \cline{3-12}
			& & FR19  & FR20 & FR21 & FR22 & FR23-25  & FR26 & FR27 & FR28 & FR29 & F30 \\ \cline{1-12}
			\multicolumn{1}{ |c| }{\multirow{11}{*}{Test Cases} } &
			\multicolumn{1}{|c| }{ FR-DS-1} &X&&&&&&&&& \\ \cline{2-12}
			\multicolumn{1}{|c| }{} 	                  &
			\multicolumn{1}{|c| }{ FR-DS-2} &&X&&&&&&&& \\ \cline{2-12}
			\multicolumn{1}{|c| }{} 	                  &
			\multicolumn{1}{|c| }{ FR-DS-3} &&&X&&&&&&& \\ \cline{2-12}
			\multicolumn{1}{|c| }{} 	                  &
			\multicolumn{1}{|c| }{ FR-DS-4} &&&&X&&&&&& \\ \cline{2-12}
			\multicolumn{1}{|c| }{} 	                  &
			\multicolumn{1}{|c| }{ FR-DS-5} &&&&&X&&&&& \\ \cline{2-12}
			\multicolumn{1}{|c| }{} 	                  &
			\multicolumn{1}{|c| }{ FR-DS-6} &&&&&&X&&&& \\ \cline{2-12}
			\multicolumn{1}{|c| }{} 	                  &
			\multicolumn{1}{|c| }{ FR-DS-7} &&&&&&&X&&& \\ \cline{2-12}
			\multicolumn{1}{|c| }{} 	                  &
			\multicolumn{1}{|c| }{ FR-DS-8} &&&&&&&&X&X& \\ \cline{1-12}
		\end{tabularx}
	\end{center}
\end{table}

\begin{table}[H]
	\begin{center}
		\caption{\textbf{Traceability Matrix for Workout Page Functional Requirements}}
		\begin{tabularx}{\textwidth}{cc|c|c|c|c|c|c|c|c|}
			\cline{3-7}
			& & \multicolumn{5}{ c|}{Requirements} \\ \cline{3-7}
			& & FR31  & FR32 & FR33 & FR34 & FR35 \\ \cline{1-7}
			\multicolumn{1}{ |c| }{\multirow{6}{*}{Test Cases} } &
			\multicolumn{1}{|c| }{ FR-WP-1} &X&&&& \\ \cline{2-7}
			\multicolumn{1}{|c| }{} 	                  &
			\multicolumn{1}{|c| }{ FR-WP-2} &&X&&& \\ \cline{2-7}
			\multicolumn{1}{|c| }{} 	                  &
			\multicolumn{1}{|c| }{ FR-WP-3} &&&X&& \\ \cline{2-7}
			\multicolumn{1}{|c| }{} 	                  &
			\multicolumn{1}{|c| }{ FR-WP-4} &&&&X& \\ \cline{2-7}
			\multicolumn{1}{|c| }{}                        &
			\multicolumn{1}{|c| }{ FR-WP-5} &&&&&X \\ \cline{1-7}
		\end{tabularx}
	\end{center}
\end{table}


\begin{table}[H]
	\begin{center}
		\caption{\textbf{Traceability Matrix for Rest Section Functional Requirements}}
		\begin{tabularx}{\textwidth}{cc|c|c|c|c|c|}
			\cline{3-4}
			& & \multicolumn{2}{ c|}{Requirements} \\ \cline{3-4}
			& & FR36  & FR37 \\ \cline{1-4}
			\multicolumn{1}{ |c| }{\multirow{3}{*}{Test Cases} } &
			\multicolumn{1}{|c| }{ FR-RS-1} &X& \\ \cline{2-4}
			\multicolumn{1}{|c| }{}                        &
			\multicolumn{1}{|c| }{ FR-RS-2} &&X \\ \cline{1-4}
		\end{tabularx}
	\end{center}
\end{table}

\begin{table}[H]
	\begin{center}
		\caption{\textbf{Traceability Matrix for Look and Feel Nonfunctional Requirements}}
		\begin{tabularx}{\textwidth}{cc|c|c|c|c|c|}
			\cline{3-4}
			& & \multicolumn{2}{ c|}{Requirements} \\ \cline{3-4}
			& & LF1  & LF2 \\ \cline{1-4}
			\multicolumn{1}{ |c| }{\multirow{3}{*}{Test Cases} } &
			\multicolumn{1}{|c| }{ NFR-LF1} &X& \\ \cline{2-4}
			\multicolumn{1}{|c| }{}                        &
			\multicolumn{1}{|c| }{ NFR-LF22} &&X \\ \cline{1-4}
		\end{tabularx}
	\end{center}
\end{table}


\begin{table}[H]
	\begin{center}
		\caption{\textbf{Traceability Matrix for Usability and Humanity Nonfunctional Requirements}}
		\begin{tabularx}{\textwidth}{cc|c|c|c|c|c|c|c|c|c|}
			\cline{3-8}
			& & \multicolumn{6}{ c|}{Requirements} \\ \cline{3-8}
			& & UH1  & UH2 & UH3 & UH4 & UH5 & UH6 \\ \cline{1-8}
			\multicolumn{1}{ |c| }{\multirow{6}{*}{Test Cases} } &
			\multicolumn{1}{|c| }{ NFR-UH1} &X&&&&& \\ \cline{2-8}
			\multicolumn{1}{|c| }{} 	                  &
			\multicolumn{1}{|c| }{ NFR-UH2} &&X&&&& \\ \cline{2-8}
			\multicolumn{1}{|c| }{} 	                  &
			\multicolumn{1}{|c| }{ NFR-UH3} &&&X&&& \\ \cline{2-8}
			\multicolumn{1}{|c| }{} 	                  &
			\multicolumn{1}{|c| }{ NFR-UH4} &&&&X&&\\ \cline{2-8}
			\multicolumn{1}{|c| }{}                        &
			\multicolumn{1}{|c| }{NFR-UH5} &&&&&X& \\ \cline{1-8}
		\end{tabularx}
	\end{center}
\end{table}

\begin{table}[H]
	\begin{center}
		\caption{\textbf{Traceability Matrix for Perfromance Nonfunctional Requirements}}
		\begin{tabularx}{\textwidth}{cc|c|c|c|c|c|c|c|c|}
			\cline{3-7}
			& & \multicolumn{5}{ c|}{Requirements} \\ \cline{3-7}
			& & PE1  & PE2 & PE3 & PE4 & PE5 \\ \cline{1-7}
			\multicolumn{1}{ |c| }{\multirow{6}{*}{Test Cases} } &
			\multicolumn{1}{|c| }{ NFR-PE1} &X&&&& \\ \cline{2-7}
			\multicolumn{1}{|c| }{} 	                  &
			\multicolumn{1}{|c| }{ NFR-PE2} &&X&&& \\ \cline{2-7}
			\multicolumn{1}{|c| }{} 	                  &
			\multicolumn{1}{|c| }{ NFR-PE3} &&&&X& \\ \cline{2-7}
			\multicolumn{1}{|c| }{} 	                  &
			\multicolumn{1}{|c| }{ NFR-PE4} &&&&&X \\ \cline{1-7}
		\end{tabularx}
	\end{center}
\end{table}

\begin{table}[H]
	\begin{center}
		\caption{\textbf{Traceability Matrix for Operational Nonfunctional Requirements}}
		\begin{tabularx}{\textwidth}{cc|c|c|c|c|c|}
			\cline{3-4}
			& & \multicolumn{2}{ c|}{Requirements} \\ \cline{3-4}
			& & OE1  & OE2 \\ \cline{1-4}
			\multicolumn{1}{ |c| }{\multirow{1}{*}{Test Cases} } &
			\multicolumn{1}{|c| }{ NFR-OE1} &X& \\ \cline{1-4}
		\end{tabularx}
	\end{center}
\end{table}

\begin{table}[H]
	\begin{center}
		\caption{\textbf{Traceability Matrix for Maintainability and Portability Nonfunctional Requirements}}
		\begin{tabularx}{\textwidth}{cc|c|c|c|c|c|c|}
			\cline{3-5}
			& & \multicolumn{3}{ c|}{Requirements} \\ \cline{3-5}
			& & MP1  & MP2 & MP3 \\ \cline{1-5}
			\multicolumn{1}{ |c| }{\multirow{2}{*}{Test Cases} } &
			\multicolumn{1}{|c| }{ NFR-MP1} &X&& \\ \cline{2-5}
			\multicolumn{1}{|c| }{}                        &
			\multicolumn{1}{|c| }{ NFR-MP2} &&X& \\ \cline{1-5}
		\end{tabularx}
	\end{center}
\end{table}

\begin{table}[H]
	\begin{center}
		\caption{\textbf{Traceability Matrix for Security Nonfunctional Requirements}}
		\begin{tabularx}{\textwidth}{cc|c|c|c|c|c|}
			\cline{3-4}
			& & \multicolumn{2}{ c|}{Requirements} \\ \cline{3-4}
			& & SE1  & SE2 \\ \cline{1-4}
			\multicolumn{1}{ |c| }{\multirow{3}{*}{Test Cases} } &
			\multicolumn{1}{|c| }{NFR-SE1} &X& \\ \cline{2-4}
			\multicolumn{1}{|c| }{}                        &
			\multicolumn{1}{|c| }{NFR-SE2} &&X \\ \cline{1-4}
		\end{tabularx}
	\end{center}
\end{table}

\begin{table}[H]
	\begin{center}
		\caption{\textbf{Traceability Matrix for Cultural and Political Nonfunctional Requirements}}
		\begin{tabularx}{\textwidth}{cc|c|c|c|c|}
			\cline{3-3}
			& & \multicolumn{1}{ c|}{Requirements} \\ \cline{3-3}
			& & CU1 \\ \cline{1-3}
			\multicolumn{1}{ |c| }{\multirow{1}{*}{Test Cases} } &
			\multicolumn{1}{|c| }{NFR-CU1} &X \\ \cline{1-3}
		\end{tabularx}
	\end{center}
\end{table}

% \section{Unit Test Description}

% \wss{Reference your MIS and explain your overall philosophy for test case
% 	selection.}  
% \wss{This section should not be filled in until after the MIS has
% 	been completed.}

% \subsection{Unit Testing Scope}

% \wss{What modules are outside of the scope.  If there are modules that are
% 	developed by someone else, then you would say here if you aren't planning on
% 	verifying them.  There may also be modules that are part of your software, but
% 	have a lower priority for verification than others.  If this is the case,
% 	explain your rationale for the ranking of module importance.}

% \subsection{Tests for Functional Requirements}

% \wss{Most of the verification will be through automated unit testing.  If
% 	appropriate specific modules can be verified by a non-testing based
% 	technique.  That can also be documented in this section.}

% \subsubsection{Module 1}

% \wss{Include a blurb here to explain why the subsections below cover the module.
% 	References to the MIS would be good.  You will want tests from a black box
% 	perspective and from a white box perspective.  Explain to the reader how the
% 	tests were selected.}

% \begin{enumerate}
	
% 	\item{test-id1\\}
	
% 	Type: \wss{Functional, Dynamic, Manual, Automatic, Static etc. Most will
% 		be automatic}
	
% 	Initial State: 
	
% 	Input: 
	
% 	Output: \wss{The expected result for the given inputs}
	
% 	Test Case Derivation: \wss{Justify the expected value given in the Output field}
	
% 	How test will be performed: 
	
% 	\item{test-id2\\}
	
% 	Type: \wss{Functional, Dynamic, Manual, Automatic, Static etc. Most will
% 		be automatic}
	
% 	Initial State: 
	
% 	Input: 
	
% 	Output: \wss{The expected result for the given inputs}
	
% 	Test Case Derivation: \wss{Justify the expected value given in the Output field}
	
% 	How test will be performed: 
	
% 	\item{...\\}
	
% \end{enumerate}

% \subsubsection{Module 2}

% ...

% \subsection{Tests for Nonfunctional Requirements}

% \wss{If there is a module that needs to be independently assessed for
% 	performance, those test cases can go here.  In some projects, planning for
% 	nonfunctional tests of units will not be that relevant.}

% \wss{These tests may involve collecting performance data from previously
% 	mentioned functional tests.}

% \subsubsection{Module ?}

% \begin{enumerate}
	
% 	\item{test-id1\\}
	
% 	Type: \wss{Functional, Dynamic, Manual, Automatic, Static etc. Most will
% 		be automatic}
	
% 	Initial State: 
	
% 	Input/Condition: 
	
% 	Output/Result: 
	
% 	How test will be performed: 
	
% 	\item{test-id2\\}
	
% 	Type: Functional, Dynamic, Manual, Static etc.
	
% 	Initial State: 
	
% 	Input: 
	
% 	Output: 
	
% 	How test will be performed: 
	
% \end{enumerate}

% \subsubsection{Module ?}

% ...

% \subsection{Traceability Between Test Cases and Modules}

% \wss{Provide evidence that all of the modules have been considered.}

\bibliographystyle{plainnat}

\bibliography{../../refs/References}

\newpage

\section{Appendix}

%This is where you can place additional information.

\subsection{Symbolic Parameters}

\noindent MINIMUM\_TEST\_SCORE = \hypertarget{MINIMUM_TEST_SCORE}{8.5}\\
MINIMUM\_TEST\_SCORE\_2 = \hypertarget{MINIMUM_TEST_SCORE_2}{9.5}\\
MAXIMUM\_ACCESS\_TIME = \hypertarget{MAXIMUM_ACCESS_TIME}{10}\\
MIN\_APPROVAL\_RATING = \hypertarget{MIN_APPROVAL_RATING}{85\%}\\
MIN\_APPROVAL\_RATING\_2 = \hypertarget{MIN_APPROVAL_RATING_2}{95\%}\\
MIN\_USER\_LOAD = \hypertarget{MIN_USER_LOAD}{50}\\
MIN\_DATA\_POINTS = \hypertarget{MIN_DATA_POINTS}{1000000}






%The definition of the test cases will call for SYMBOLIC\_CONSTANTS.
%Their values are defined in this section for easy maintenance.

\subsection{Usability Survey Questions?}

\begin{enumerate}
	\item How would you rate overall neatness of UI/UX design of REVITALIZE out of 10? What are elements you like related to neatness? What are elements you would improve related to neatness?
	\item Are colours used in REVITALIZE non-intrusive, appealing and/or contrasting? Rate each factor of non-intrusive, appealing and contrasting out of 10? What are elements you like that elevate these factors? What are elements you would improve to elevate these factors?
	\item How would you rate the overall ability to use REVITALIZE with only one hand/one finger out of 10? What are elements that help with this? What are elements you would improve?
\end{enumerate}

\section{Reflection Appendix}

\subsection{Knowledge and Skills Needed}

\noindent Bill Nguyen: My primary role for VnV plan is to be a back-end developer tester, so some skills needed would be how to write proper and effective unit tests for back-end implementation of project, integration tests that tests the project end to end and with the front-end, and also maybe some tests for performance such as load, stress test etc. for back-end code.\\

\noindent Syed Bokhari: My primary role for VnV plan is to be a Database Application tester. Some skills needed would be how to configure API testing environments and database query accuracy. I must know how to prepare the testing environment, writeunit tests for SQL queries and configure test data to ensure the accuracy of the results.\\

\noindent Mahmoud Anklis: My primary role for the VnV plan is to be a System Integration tester. Some of the skills needed would be how to setup system tests that test the integration of all modules. I would also have to know how to write scripts that test the interfaces of each component and ensure the entire system is behaving properly.\\

\noindent Hasan Kibria: My primary role for the VnV plan is to be a Frontend Tester. Some of the skills needed would be knowledge of frontend testing frameworks for our language of choice, as well as knowing how to properly mock unnecessary resources (i.e. the backend) to truly test targeted frontend functionality in the manner we aim to.\\

\noindent Logan Brown: My primary role for the VnV plan is to be a Performance Tester. First I would need to know what tools are available for monitoring the performance of our app and then the skill I would need to learn is how to properly analyze the measured metrics. I would also have to know how to simulate performance on a variety of devices with different capabilities.\\

\subsection{How Knowledge and Skills Will Be Acquired}

\noindent Bill Nguyen: Knowledge and skills will be acquired by researching techniques on how to write effective unit tests for the back-end for example using the "arrange act assert" and potentially setting minimum thresholds for code/branch coverage to improve consistency and organization of tests. For integration and performance tests also perform research to find effective ways to test back-end with all aspects of project and applying it often.\\

\noindent Syed Bokhari: The required knowledge and skills will be acquired by setting up practice environments via virtual machines and testing the API implementation with smaller scale applications. Once I can confirm the accuracy of the database queries and API requests with unit tests, I can configure a larger portion of the code to ingest the data from the API and store in the database.\\

\noindent Mahmoud Anklis: The  knowledge and skills required for system integration testing will be acquired by understanding the software architecture and how each component communicates with the other. I will also have to research methods on how to build scripts the can test the entire application's components (Front-End, Back-End, Database etc), where the script is able to pin point the component that led to a failure if there is a failure in when testing.\\

\noindent Hasan Kibria: I hope to acquire this knowledge by reading online docs, looking at code examples, and trying my best to re-emulate tests I find on the internet to fit our specific needs. Once I find myself to be familiar with the technologies and infrastructure at hand, then I would continue to learn by diving deeper into testing principles and following through the targeted test implementation.\\

\noindent Logan Brown: To learn the skills I outlined above I would need to research different performance measuring tools and what they measure using online documentation. Most likely I could use a tool that is a plugin for the development platform we will be using. I will also learn these skills by measuring the performance of simulated versions of the application, making changes to the implementation and seeing the effect these changes have on the performance.\\

\end{document}