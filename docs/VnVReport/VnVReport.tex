\documentclass[12pt, titlepage]{article}

\usepackage{booktabs}
\usepackage{tabularx}
\usepackage{hyperref}
\hypersetup{
colorlinks,
citecolor=black,
filecolor=black,
linkcolor=red,
urlcolor=blue
}
\usepackage[round]{natbib}

\usepackage{titlesec}
\usepackage{placeins}
\usepackage{graphicx}
\usepackage{xkeyval}
\usepackage[dvipsnames]{xcolor} % for different colour comments
\usepackage{tabto}
\usepackage{mdframed}
\usepackage{lscape}
\usepackage{multirow}
\usepackage{float}
\newcommand{\comma}{,}

%% Comments

\usepackage{color}

\newif\ifcomments\commentstrue %displays comments
%\newif\ifcomments\commentsfalse %so that comments do not display

\ifcomments
\newcommand{\authornote}[3]{\textcolor{#1}{[#3 ---#2]}}
\newcommand{\todo}[1]{\textcolor{red}{[TODO: #1]}}
\else
\newcommand{\authornote}[3]{}
\newcommand{\todo}[1]{}
\fi

\newcommand{\wss}[1]{\authornote{blue}{SS}{#1}} 
\newcommand{\plt}[1]{\authornote{magenta}{TPLT}{#1}} %For explanation of the template
\newcommand{\an}[1]{\authornote{cyan}{Author}{#1}}


\makeatletter
\define@cmdkey      [TP] {test}     {name}       {}
\define@cmdkey      [TP] {test}     {desc}       {}
\define@cmdkey      [TP] {test}     {type}       {}
\define@cmdkey      [TP] {test}     {init}       {}
\define@cmdkey      [TP] {test}     {input}      {}
\define@cmdkey      [TP] {test}     {output}     {}
\define@cmdkey      [TP] {test}     {pass}       {}
\define@cmdkey      [TP] {test}     {user}       {}
\define@cmdkey      [TP] {test}     {result}     {}

\newcounter{TestNum}
\newcommand{\testauto}[1]{
\setkeys[TP]{test}{#1}
\refstepcounter{TestNum}
\begin{mdframed}[linewidth=1pt]
\begin{tabularx}{\textwidth}{@{}p{3cm}X@{}}
	{\bf Test \#\theTestNum:} & {\bf \cmdTP@test@name}\\[\baselineskip]
	{\bf Description:} & \cmdTP@test@desc\\[0.5\baselineskip]
	{\bf Type:} & \cmdTP@test@type\\[0.5\baselineskip]
	{\bf Initial State:} & \cmdTP@test@init\\[0.5\baselineskip]
	{\bf Input:} & \cmdTP@test@input\\[0.5\baselineskip]
	{\bf Output:} & \cmdTP@test@output\\[0.5\baselineskip]
	{\bf Expected:} & \cmdTP@test@pass\\[\baselineskip]
	{\bf Result:} & \cmdTP@test@result
\end{tabularx}
\end{mdframed}
}

\newcommand{\testautob}[1]{
\setkeys[TP]{test}{#1}
\refstepcounter{TestNum}
\begin{mdframed}[linewidth=1pt]
\begin{tabularx}{\textwidth}{@{}p{3cm}X@{}}
	{\bf Test \#\theTestNum:} & {\bf \cmdTP@test@name}\\[\baselineskip]
	{\bf Description:} & \cmdTP@test@desc\\[0.5\baselineskip]
	{\bf Type:} & \cmdTP@test@type\\[0.5\baselineskip]
	{\bf Pass:} & \cmdTP@test@pass\\[\baselineskip]
	{\bf Result:} & \cmdTP@test@result
\end{tabularx}
\end{mdframed}
}

\newcommand{\testmanual}[1]{
\setkeys[TP]{test}{#1}
\refstepcounter{TestNum}
\begin{mdframed}[linewidth=1pt]
\begin{tabularx}{\textwidth}{@{}p{3cm}X@{}}
	{\bf Test \#\theTestNum:} & {\bf \cmdTP@test@name}\\[\baselineskip]
	{\bf Description:} & \cmdTP@test@desc\\[0.5\baselineskip]
	{\bf Type:} & \cmdTP@test@type\\[0.5\baselineskip]
	{\bf Tester(s):} & \cmdTP@test@user\\[0.5\baselineskip]
	{\bf Pass:} & \cmdTP@test@pass\\[\baselineskip]
	{\bf Result:} & \cmdTP@test@result
\end{tabularx}
\end{mdframed}
}

\begin{document}

\title{Verification and Validation Report: REVITALIZE} 
\author{Team 13, 
\\ Bill Nguyen
\\ Syed Bokhari
\\ Hasan Kibria
\\ Mahmoud Anklis
\\ Youssef Dahab
\\ Logan Brown}
\date{\today}

\maketitle

\pagenumbering{roman}

\section{Revision History}

\begin{tabularx}{\textwidth}{p{3cm}p{2cm}X}
\toprule {\bf Date} & {\bf Version} & {\bf Notes}\\
\midrule
March 5th, 2023 & Bill Nguyen & Adding Unit Tests for Workout and Rest Section\\
March 5th, 2023 & Youssef Dahab & Added Functional Requirements Evaluation\\
March 6th, 2023 & Youssef Dahab & Added Changes Due To Testing\\
March 8th, 2023 & Hasan Kibria & Adding Unit Tests for Diet Section\\
March 8th, 2023 & Youssef Dahab & Added Reflection\\
March 8th, 2023 & Logan Brown & Added Non-Functional Requirements Evaluation\\
March 8th, 2023 & Mahmoud Anklis & Added Unit Tests for the User Section and Reflection\\
\bottomrule
\end{tabularx}

~\newpage

\section{Symbols, Abbreviations and Acronyms}

\renewcommand{\arraystretch}{1.2}
\begin{tabular}{l l} 
\toprule		
\textbf{symbol} & \textbf{description}\\
\midrule 
REVITALIZE & Name of application\\
SRS & Software Requirements Specification\\
VnV & Verification and Validation\\
FR & Functional Requirement\\
NFR & Non Functional Requirement\\
LP & Login Page\\
SP & Sign-up Page\\
MP & Main Page or Maintainability and Portability Requirements\\
DS & Diet Section\\
WS & Workout Section\\
RS & Rest Section\\
LF & Look and Feel Requirements\\
UH & Usability and Humanity Requirements\\
PE & Performance Requirement\\
OE & Operational Requirement\\
SE & Security Requirement\\
CU & Cultural Requirement\\
\bottomrule
\end{tabular}\\

\subsection{Symbolic Parameters}

\noindent MINIMUM\_TEST\_SCORE = \hypertarget{MINIMUM_TEST_SCORE}{8.5}\\
MINIMUM\_TEST\_SCORE\_2 = \hypertarget{MINIMUM_TEST_SCORE_2}{9.5}\\
MAXIMUM\_ACCESS\_TIME = \hypertarget{MAXIMUM_ACCESS_TIME}{10}\\
MIN\_APPROVAL\_RATING = \hypertarget{MIN_APPROVAL_RATING}{85\%}\\
MIN\_APPROVAL\_RATING\_2 = \hypertarget{MIN_APPROVAL_RATING_2}{95\%}\\
MIN\_USER\_LOAD = \hypertarget{MIN_USER_LOAD}{50}\\
MIN\_DATA\_POINTS = \hypertarget{MIN_DATA_POINTS}{1000000}

\newpage

\tableofcontents

\listoftables %if appropriate

\listoffigures %if appropriate

\newpage

\pagenumbering{arabic}

This document details the complete testing process for REVITALIZE, as laid out in the project test plan. It contains an evaluation of the project’s functional and non-functional requirements that are defined in the \hyperlink{https://github.com/BillNguyen1999/REVITALIZE/blob/main/docs/SRS/SRS.pdf}{\textbf{SRS}}, the changes made due to testing, and an analysis of the traceability between requirements and modules.

\section{Functional Requirements Evaluation}

\subsection{Login Page}
\testauto{
name = {FR-LP-1},
desc = Testing that login page is displayed upon starting the application,
type = Manual,
init = Loading stage of the login page,
input = An event that loads the login page,
output = Login page is displayed with all necessary components,
pass = ,
result = \textcolor{Green}{PASS}
}
\testauto{
name = {FR-LP-2},
desc = Testing that login page displays fillable username textbox,
type = Manual,
init = Login page is displayed with username textbox,
input = Enter username information in textbox,
output = Username information entered is displayed in textbox,
pass = ,
result = \textcolor{Green}{PASS}
}
\testauto{
name = {FR-LP-3},
desc = Testing that login page displays fillable password textbox,
type = Manual,
init = Login page is displayed with password textbox,
input = Enter password information in textbox,
output = Password information entered is displayed in textbox via hidden text,
pass = ,
result = \textcolor{Green}{PASS}
}
\testauto{
name = {FR-LP-4},
desc = Testing that login page displays login button,
type = Manual,
init = Login page is displayed with login button,
input = Click the login button,
output = User logs in after the system checks the validity of the input parameters in the login page,
pass = Login button is displayed and user logged in successfully,
result = \textcolor{Green}{PASS}
}
\testauto{
name = {FR-LP-5},
desc = Testing that login page displays forgot password button,
type = Manual,
init = Login page is displayed with forgot password button,
input = Click forgot password button,
output = Display forgot password screen with textbox to enter email,
pass = ,
result = \textcolor{Green}{PASS}
}
\testauto{
name = {FR-LP-6},
desc = Testing that login page displays a stay looged in checkbox,
type = Manual,
init = Login page is displayed with stay logged in checkbox that is empty,
input = Click stay logged in checkbox,
output = Display a check-mark in the stay logged in checkbox if checkbox is empty. Else if checkbox contains check-mark already it will then display an empty checkbox,
pass = ,
result = \textcolor{Green}{PASS}
}
\testauto{
name = {FR-LP-7},
desc = Testing that application saves prior login information if stay logged in checkbox is checked,
type = Manual,
init = Loading stage of REVITALIZE where previous state had stay logged in
checkbox checked,
input = An event that loads REVITALIZE,
output = Display main page \comma{} with same data from previous state of main page,
pass = ,
result = \textcolor{Green}{PASS}
}
\testauto{
name = {FR-LP-8},
desc = Testing that login page displays sign-up button that redirects to sign-up page,
type = Manual,
init = Login page is displayed with sign up button,
input = Click sign up button,
output = Loads and displays sign up page,
pass = ,
result = \textcolor{Green}{PASS}
}
\testauto{
name = {FR-LP-9},
desc = Testing if application checks validity of input parameters in login page,
type = Manual,
init = Login page is displayed with inputted information in username and pass-
word text-boxes,
input = Click login button,
output = If failure state\comma{} display an invalid password or username banner. Else if success state\comma{} load and display main page,
pass = ,
result = \textcolor{Green}{PASS}
}

\subsection{Signup Page}
\testauto{
name = {FR-SP-1},
desc = Testing that signup page displays fillable username textbox,
type = Manual,
init = Signup page is displayed with username textbox,
input = Enter username information in textbox,
output = Username information entered is displayed in textbox,
pass = ,
result = \textcolor{Green}{PASS}
}
\testauto{
name = {FR-SP-2},
desc = Testing that signup page displays fillable password textbox,
type = Manual,
init = Signup page is displayed with password textbox,
input = Enter password information in textbox,
output = Password information entered is displayed in textbox via hidden text,
pass = ,
result = \textcolor{Green}{PASS}
}
\testauto{
name = {FR-SP-3},
desc = Testing that signup page displays fillable email textbox,
type = Manual,
init = Signup page is displayed with email textbox,
input = Enter email information in textbox,
output = Email information entered is displayed in textbox,
pass = ,
result = \textcolor{Green}{PASS}
}
\testauto{
name = {FR-SP-4},
desc = Testing that signup page displays signup button,
type = Manual,
init = Signup page is displayed with signup button,
input = Click the signup button,
output = User signs up after the system checks the validity of the input parameters on the signup page,
pass = ,
result = \textcolor{Green}{PASS}
}
\testauto{
name = {FR-SP-5},
desc = Testing if application checks validity of input parameters in signup page,
type = Manual,
init = Signup page is displayed with inputted information in username and
password text-boxes,
input = Click signup button,
output = If failure state then display an invalid username/password or email banner. Else if success state then load and display login page,
pass = ,
result = \textcolor{Green}{PASS}
}

\subsection{Main Page}
\testauto{
name = {FR-MP-1},
desc = Testing that the application displays a calendar with current date on successful login,
type = Manual,
init = Main page is displayed with calender of current date,
input = An event that loads the main page,
output = Main page is displayed with all necessary components,
pass = ,
result = \textcolor{Green}{PASS}
}
\testauto{
name = {FR-MP-2},
desc = Testing that the application has a previous day and a next day button on each page after successful login,
type = Manual,
init = Main page and Diet\comma{} Workout\comma{} Rest sections are displayed with previous day and next day buttons,
input = An event that loads the main page\comma{} Diet\comma{} Workout\comma{} Rest sections and the previous day and next day buttons are clicked,
output = Main page\comma{} Diet\comma{} Workout\comma{} Rest sections are displayed with previous day and next day buttons. Once the next day button is clicked\comma{} the calendar refreshes the calendar information for the next day. Once the previous day button is clicked\comma{} the calendar refreshes the calendar information for the previous day,
pass = ,
result = \textcolor{Green}{PASS}
}
\testauto{
name = {FR-MP-3},
desc = Testing that a back button is displayed on each user interface after a section is selected,
type = Manual,
init = Each interaction after leaving the main page must have a visible back button,
input = An event that loads the next user interface after leaving the main page and the back button is clicked,
output = The next user interface after leaving the main page is displayed with a back button. Once the back button is clicked the main page is loaded,
pass = ,
result = \textcolor{Green}{PASS}
}
\testauto{
name = {FR-MP-4},
desc = Testing that the application displays the sections Diet\comma{} Exercise\comma{} and Rest on the current calendar day,
type = Manual,
init = Main page is displayed with Diet\comma{} Exercise and Rest buttons available to click,
input = An event that loads the main page and the Diet\comma{} Exercise and Rest buttons are clicked,
output = Main page is displayed with Diet\comma{} Exercise and Rest buttons. If the Diet button is clicked\comma{} the Diet interface is loaded. If the Exercise button is clicked\comma{} the Exercise interface is loaded. If the Rest button is clicked\comma{} the Rest interface is loaded,
pass = ,
result = \textcolor{Green}{PASS}
}

\subsection{Diet Section Page}
\testauto{
name = {FR-DS-1},
desc = Testing that application prompts the user to height\comma{} input dietary\comma{} weight\comma{} calorie information on initial launch of Diet section,
type = Manual,
init = Diet section is initialized for the first time and an initial information dialog is launched,
input = An event that loads the diet section for the first time,
output = A fillable dialog box is launched with height\comma{} dietary information\comma{} weight and calorie information,
pass = ,
result = \textcolor{Green}{PASS}
}
\testauto{
name = {FR-DS-2},
desc = Testing that the application saves initial user height\comma{} dietary\comma{} weight\comma{} calorie information,
type = Manual,
init = Diet section is initialized for the first time and an initial information dialog is launched,
input = Initial information dialog values are filled,
output = Initial information values are saved to the database,
pass = ,
result = \textcolor{Green}{PASS}
}
\testauto{
name = {FR-DS-3},
desc = Testing that the application initializes with a list of food logged on the current calendar day,
type = Manual,
init = Section is initialized with a list of food logged for the current calendar day,
input = An event that loads the rest section,
output = A list of inputted food is loaded for the current calendar day,
pass = ,
result = \textcolor{Green}{PASS}
}
\testauto{
name = {FR-DS-4},
desc = Testing that Diet section displays add food button,
type = Manual,
init = Diet section is displayed with add food button,
input = Click add food button,
output = A user interface is launched that lets the user select between searching for food or adding a custom meal,
pass = ,
result = \textcolor{Green}{PASS}
}
\testauto{
name = {FR-DS-5},
desc = Testing that Diet section displays search food button,
type = Manual,
init = Food adding user interface is displayed with search food button,
input = Click the search food button,
output = A recipe criteria user interface is launched that displays a list of modifiable criteria and a search button,
pass = ,
result = \textcolor{Green}{PASS}
}
\testauto{
	name = {FR-DS-6},
	desc = Testing that search food button launches recipe criteria user interface,
	type = Manual,
	init = Recipe criteria user interface is launched,
	input = Search criteria is modified and search button is clicked,
	output = List of recipes are loaded correctly based on constraints of search criteria,
	pass = ,
	result = \textcolor{Green}{PASS}
}
\testauto{
name = {FR-DS-7},
desc = Testing that recipe search displays correct recipe values based on input constraints,
type = Manual,
init = Recipe list is loaded based on search constraints,
input = Add recipe button is clicked,
output = Selected recipe is added to the list of food logged on the current calendar day,
pass = ,
result = \textcolor{Green}{PASS}
}
\testauto{
name = {FR-DS-8},
desc =  Testing that add custom meal button adds meal to list of logged food for the current calendar day upon filling necessary recipe information textboxes,
type = Manual,
init = Food adding interface is displayed with add custom meal button,
input = Click add custom meal button,
output = A dialog box is launched that lets the user fill in custom meal information. The meal is added to the food log list of the current calendar day,
pass = ,
result = \textcolor{Green}{PASS}
}

\subsection{Workout Section Page}
\testauto{
name = {FR-WS-1},
desc = Testing that the Workout section initializes with a preset list of exercises on the current calendar day,
type = Manual,
init = Workout section is initialized with a preset list of exercises of the current calendar day,
input = An event that loads the workout section,
output = A preset list of exercises is loaded for the current calendar day,
pass = ,
result = \textcolor{Green}{PASS}
}
\testauto{
	name = {FR-WS-2},
	desc = Testing that the Workout section has add exercise button,
	type = Manual,
	init = Workout section is displayed with add exercise button,
	input = Click add exercise button,
	output = A dialog box is launched that lets the user fill custom exercise information. The exercise is added to the exercise list of the current calendar day,
	pass = ,
	result = \textcolor{Green}{PASS}
}
\testauto{
name = {FR-WS-3},
desc = Testing that the Workout section has delete exercise button,
type = Manual,
init = Each exercise in the workout section is displayed with a delete exercise button,
input = Click delete exercise button,
output = The exercise is deleted from the exercise list of the current calendar day,
pass = ,
result = \textcolor{Green}{PASS}
}
\testauto{
name = {FR-WS-4},
desc = Testing that exercises display an edit exercise button that launches the changeable exercise information when clicked,
type = Manual,
init = Each exercise in the workout section is displayed with an edit exercise button,
input = click edit exercise button,
output = A fillable dialog box is launched with information of the exercise. Once
the edit exercise button is clicked the dialog box will close and update the exercise
information in the list of exercises for the current calendar day,
pass = ,
result = \textcolor{Green}{PASS}
}
\testauto{
name = {FR-WS-5},
desc = Testing that the Workout section prompts the user to add repetitions and sets of each exercise logged in the current calendar day,
type = Manual,
init = Workout section is displayed with list of exercises for current calendar day,
input = An event that loads the workout section,
output = If repetition and sets for exercises not logged then dialog box for exercise is launched and the missing repetition and set values are highlighted,
pass = ,
result = \textcolor{Green}{PASS}
}

\subsection{Rest Section Page}
\testauto{
name = {FR-RS-1},
desc = Testing that Rest section launches with sleep statistics of current calendar day,
type = Manual,
init = Rest section is initialized with the sleep statistics of the current calendar day,
input = An event that loads the rest section,
output = Sleep statistics are loaded for the current calendar day,
pass = ,
result = \textcolor{Green}{PASS}
}
\testauto{
name = {FR-RS-2},
desc = Testing that user can alter inaccurate sleep data,
type = Manual,
init = Rest section is initialized with the sleep statistics of the current calendar day,
input = Alter sleep data,
output = The sleep data is updated with user changes,
pass = ,
result = \textcolor{Green}{PASS}
}

\section{Nonfunctional Requirements Evaluation}

\subsection{Look and Feel}

\testmanual{
	name = {NFR-LF1},
	desc = Testing that UI/UX elements are displayed neatly and correctly,
	type = Manual,
	user = Stakeholders,
	pass = Average Q1 survey score of at least \hyperlink{MINIMUM_TEST_SCORE}{MINIMUM\_TEST\_SCORE},
	result = \textcolor{Green}{PASSED} with an agreement of 8.8 out of 10
}

\testmanual{
	name = {NFR-LF2},
	desc = Testing that colours used are acceptable,
	type = Manual,
	user = Stakeholders,
	pass = Average Q2 survey score of at least \hyperlink{MINIMUM_TEST_SCORE}{MINIMUM\_TEST\_SCORE},
	result = \textcolor{Green}{PASSED} with an agreement of 10 out of 10
}

\subsection{Usability and Humanity}

\testmanual{
	name = {NFR-UH1},
	desc = Testing accessibility of application using navigation ability with one finger,
	type = Manual,
	user = Stakeholders,
	pass = Average survey score of at least \hyperlink{MINIMUM_TEST_SCORE_2}{MINIMUM\_TEST\_SCORE\_2},
	result = \textcolor{Green}{PASSED} with an agreement of 10 out of 10
}

\testmanual{
	name = {NFR-UH2},
	desc = Testing navigation speed between screens,
	type = Manual,
	user = Stakeholders,
	pass = Average survey score of at least \hyperlink{MAXIMUM_ACCESS_TIME}{MAXIMUM\_ACCESS\_TIME},
	result = \textcolor{Green}{PASSED} with an agreement of 10 out of 10
}

\testmanual{
	name = {NFR-UH3},
	desc = Testing overall accessibility through average survey result,
	type = Manual,
	user = Stakeholders,
	pass = Average survey score of at least \hyperlink{MINIMUM_TEST_SCORE}{MINIMUM\_TEST\_SCORE},
	result = \textcolor{Green}{PASSED} with an agreement of 9.3
}

\testmanual{
	name = {NFR-UH4},
	desc = Testing learnability of application,
	type = Manual,
	user = Stakeholders,
	pass = \hyperlink{MIN_APPROVAL_RATING}{MIN\_APPROVAL\_RATING} of stakeholders understand functionality in 3 iterations or less,
	result = \textcolor{Green}{PASS}
}

\testmanual{
	name = {NFR-UH5},
	desc = Testing consistency of UI,
	type = Manual,
	user = Stakeholders,
	pass = Average survey score of at least \hyperlink{MINIMUM_TEST_SCORE}{MINIMUM\_TEST\_SCORE},
	result = \textcolor{Green}{PASSED} with an agreement of 9 out of 10
}

\subsection{Performance}

\testmanual{
	name = {NFR-PE1},
	desc = Testing load times of API responses and outputs,
	type = Manual,
	user = Developers,
	pass = Load times below 5 seconds,
	result = \textcolor{Green}{PASS}
}

\testmanual{
	name = {NFR-PE2},
	desc = Testing accuracy of calculated values that contain data/numbers,
	type = Manual,
	user = Developers,
	pass = ,
	result = \textcolor{Green}{PASS}
}

\testmanual{
	name = {NFR-PE3},
	desc = Testing system performance under high load,
	type = Manual,
	user = Developers,
	pass = Previous metrics still pass with \hyperlink{MIN_USER_LOAD}{MIN\_USER\_LOAD} users,
	result = \textcolor{Green}{Tentative Pass}
}

\testmanual{
	name = {NFR-PE4},
	desc = Testing system performance with large amounts of user data,
	type = Manual,
	user = Developers,
	pass = Previous metrics still pass with \hyperlink{MIN_DATA_POINTS}{MIN\_DATA\_POINTS} per user,
	result = \textcolor{Green}{PASS}
}

\subsection{Operational}

\testmanual{
	name = {NFR-OE1},
	desc = Testing if all features are loaded with stable internet connection,
	type = Manual,
	user = Developers,
	pass = ,
	result = \textcolor{Green}{PASS}
}

\subsection{Maintainability and Portability}

\testmanual{
	name = {NFR-MP1},
	desc = Testing maintainability through cross referencing developer comments,
	type = Manual,
	user = Developers,
	pass = ,
	result = \textcolor{Green}{PASS}
}

\subsection{Security}

\testmanual{
	name = {NFR-SE1},
	desc = Testing that passwords are hashed and user data is secure,
	type = Manual,
	user = Developers,
	pass = ,
	result = \textcolor{Green}{PASS}
}

\testmanual{
	name = {NFR-SE2},
	desc = Testing that emails can only have 1 associated account,
	type = Manual,
	user = Developers,
	pass = ,
	result = \textcolor{Green}{PASS}
}

\subsection{Cultural and Political}

\testmanual{
	name = {NFR-CU1},
	desc = Testing that the displayed language is in English,
	type = Manual,
	user = Developers,
	pass = ,
	result = \textcolor{Green}{PASS}
}


\section{Comparison to Existing Implementation}	

\noindent N/A

\section{Unit Testing}

\subsection{Workout Section}
\newpage

Unit tests for the workout section: \url{https://github.com/BillNguyen1999/REVITALIZE/blob/main/src/SERVER/backend/test/exercise.test.js}.
\begin{table}[h]
\centering
\small
\begin{tabularx}{\textwidth}{|X|X|p{3cm}|p{2.5cm}|p{2.5cm}|X|}
	\hline
	Test ID & FR & Inputs & Expected Values & Actual Values & Result \\
	\hline
	WS1 & FR-WS-1 and FR-WS-5 & \{email: 'test@gmail.com', dateAdded: '2022-01-01'\}  & [{ name: 'Exercise 1' }, { name: 'Exercise 2' }] & [{ name: 'Exercise 1' }, { name: 'Exercise 2' }] & \textcolor{Green}{PASS} \\
	\hline
	WS2 & FR-WS-1 and FR-WS-5 & \{email: 'fail@gmail.com', 'dateAdded: 2022-01-01'\}  & 'Error in getting exercise list' & 'Error in getting exercise list' & \textcolor{Green}{PASS} \\
	\hline
	WS3 & FR-WS-2 & \multicolumn{3}{p{8cm}|}{\centering \{ \textcolor{Green}{success: true, message: 'Success in adding exercise data'}, id: 'exerciseid', email: 'test@gmail.com', name: 'Test Exercise', sets: 3, repetitions: 10, weight: 50, dateAdded: '2022-03-07'\}}  &  \textcolor{Green}{PASS}\\
	\hline
	WS4 & FR-WS-3 & \{email: 'test@gmail.com', dateAdded: '2022-01-01', name:'push-ups'\}  & \{success: true, message: 'Success in deleting exercise data'\} & \{success: true, message: 'Success in deleting exercise data'\} & \textcolor{Green}{PASS} \\
	\hline
	WS5 & FR-WS-3 & \{email: 'notfound@gmail.com', dateAdded: '2022-01-01', name:'push-ups'\}  & \{success: false, message: 'Was not able to delete selected exercise data'\} & \{success: false, message: 'Was not able to delete selected exercise data'\} & \textcolor{Green}{PASS} \\
	\hline
\end{tabularx}
\caption{Workout Section Unit Tests Part 1}
\label{table:workout-unit-tests}
\end{table}

\newpage

\begin{table}[h]
\centering
\small
\begin{tabularx}{\textwidth}{|X|X|p{3cm}|p{2.5cm}|p{2.5cm}|X|}
	\hline
	Test ID & FR & Inputs & Expected Values & Actual Values & Result \\
	\hline
	WS6 & FR-WS-4 & params: \{ email: 'example@gmail.com', dateAdded: '2022-01-01', name: 'exerciseName' \},
	body: \{ reps: 10, sets: 3 \}  & \{success: true, message: 'Success in editing exercise data'\} & \{success: true, message: 'Success in editing exercise data'\} & \textcolor{Green}{PASS} \\
	\hline
	WS7 & FR-WS-4 & params: \{ email: 'notfound@gmail.com', dateAdded: '2022-03-07', name: 'push-ups' \},
	body: \{ sets: 3, reps: 10\}  & \{success: false, message: "Was not able to find appropriate exercise data to edit" \} & \{success: false, message: "Was not able to find appropriate exercise data to edit" \} & \textcolor{Green}{PASS} \\
	\hline
	WS8 & FR-WS-1 and FR-WS-5 & \{email: test@gmail.com, name:'pushup', dateAdded: 2022-01-01\}  & \{success: true,
	message: 'Success in getting exercise data' \} & \{success: true, message: 'Success in getting exercise data' \} & \textcolor{Green}{PASS} \\
	\hline
	WS9 & FR-WS-1 and FR-WS-5 & \{email: 'test@gmail.com', name: 'pushup', dateAdded: 'invalid-date' \} & \{success: false, message: 'Error in getting exercise data' \} & \{success: false, message: 'Error in getting exercise data' \} & \textcolor{Green}{PASS} \\
	\hline
\end{tabularx}
\caption{Workout Section Unit Tests Part 2}
\label{table:workout-unit-tests2}
\end{table}

\newpage

\subsection{Rest Section}

Unit tests for the rest section: \url{https://github.com/BillNguyen1999/REVITALIZE/blob/main/src/SERVER/backend/test/sleep.test.js}.

\begin{table}[h]
\centering
\small
\begin{tabularx}{\textwidth}{|X|X|p{3cm}|p{2.5cm}|p{2.5cm}|X|}
	\hline
	Test ID & FR & Inputs & Expected Values & Actual Values & Result \\
	\hline
	RS1 & FR-RS-1 and FR-RS-2 & \{email: 'test@gmail.com', dateAdded: '2022-01-01'\}  & \{success: true, message: 'Success in getting sleep data'\} & \{success: true, message: 'Success in getting sleep data'\} & \textcolor{Green}{PASS} \\
	\hline
	RS2 & FR-RS-1 and FR-RS-2 & \{email: 'test@gmail.com', dateAdded: 'invalid-date'\}  & \{success: false, message: 'Error in getting sleep data'\} & \{success: false, message: 'Error in getting sleep data'\} & \textcolor{Green}{PASS} \\
	\hline
	RS3 & FR-RS-1 & \multicolumn{3}{p{8cm}|}{\centering \{ \textcolor{Green}{success: true, message: 'Success in adding sleep data'}, id: 'sleepid', email: 'test@gmail.com', sleepHour: 12, bedHour: 10, sleepMinute: 5, bedMinute: 5, dateAdded: '2022-03-07'\}}  &  \textcolor{Green}{PASS}\\
	\hline
	RS4 & FR-RS-2 & \{email: 'test@gmail.com', dateAdded: '2022-01-01'\}  & \{success: true, message: 'Success in deleting sleep data'\} & \{success: true, message: 'Success in deleting sleep data'\} & \textcolor{Green}{PASS} \\
	\hline
	RS5 & FR-RS-2 & \{email: 'notfound@gmail.com', dateAdded: '2022-01-01'\}  & \{success: false, message: 'Was not able to delete selected sleep data'\} & \{success: false, message: 'Was not able to delete selected sleep data'\} & \textcolor{Green}{PASS} \\
	\hline
\end{tabularx}
\caption{Rest Section Unit Tests Part 1}
\label{table:rest-unit-tests}
\end{table}

\newpage

\begin{table}[h]
\centering
\small
\begin{tabularx}{\textwidth}{|X|X|p{3cm}|p{2.5cm}|p{2.5cm}|X|}
	\hline
	Test ID & FR & Inputs & Expected Values & Actual Values & Result \\
	\hline
	RS6 & FR-RS-2 & params:\{ email: 'example@gmail.com', dateAdded: '2022-01-01'\},
	body: \{ sleepHour: 12, bedHour: 11, sleepMinute: 57, bedMinute: 47\}  & \{success: true, message: 'Success in editing sleep data'\} & \{success: true, message: 'Success in editing sleep data'\} & \textcolor{Green}{PASS} \\
	\hline
	RS7 & FR-RS-2 & params: \{ email: 'notfound@gmail.com', dateAdded: '2022-03-07'\},
	body: \{ sleepHour: 12, bedHour: 11, sleepMinute: 57, bedMinute: 47\}  & \{success: false, message: "Was not able to find appropriate sleep data to edit" \} & \{success: false, message: "Was not able to find appropriate sleep data to edit" \} & \textcolor{Green}{PASS} \\
	\hline
\end{tabularx}
\caption{Rest Section Unit Tests Part 2}
\label{table:rest-unit-tests2}
\end{table}

\newpage

\subsection{Diet Section}

Unit tests for the rest section: \url{https://github.com/BillNguyen1999/REVITALIZE/blob/main/src/SERVER/backend/test/foodLog.test.js}.

\begin{table}[h]
\centering
\small
\begin{tabularx}{\textwidth}{|X|X|p{3cm}|p{2.5cm}|p{2.5cm}|X|}
	\hline
	Test ID & FR & Some Inputs & Some Expected Values & Corresponding Actual Values & Result \\
	\hline
	DS1 & FR-DS-3 & \{email: 'test@gmail.com', foodDate: 2022-03-08\}  & \{success: true, message: 'Success in getting food log'\} & \{success: true, message: 'Success in getting food log'\} & \textcolor{Green}{PASS} \\
	\hline
	DS2 & FR-DS-2 & \{email: 'test@gmail.com', foodDate: 2022-03-08, calories: 1\}  & \{success: true, message: 'Meal successfully added', calories: 1\} & \{success: true, message: 'Meal successfully added', calories: 1\} & \textcolor{Green}{PASS} \\
	\hline
	DS3 & FR-DS-3 & \{email: 'test@gmail.com', foodDate: 2022-03-08, foodName: 'name'\}  & \{success: true, message: 'Success in deleting meal'\} & \{success: true, message: 'Success in deleting meal''\} & \textcolor{Green}{PASS} \\
	\hline
	DS4 & FR-RS-8 & \{email: 'test@gmail.com', foodDate: 2022-03-08\}  & \{success: true, message: 'Success in updating meal'\} & \{success: true, message: 'Success in updating meal'\} & \textcolor{Green}{PASS} \\
	\hline
\end{tabularx}
\caption{Diet Section Unit Tests Part 1}
\label{table:diet-unit-tests}
\end{table}

\newpage

\subsection{User Section}

Unit tests for the User section: \url{https://github.com/BillNguyen1999/REVITALIZE/blob/main/src/SERVER/backend/test/user.test.js}.
\begin{table}[h]
\centering
\small
\begin{tabularx}{\textwidth}{|X|X|p{3cm}|p{2.5cm}|p{2.5cm}|X|}
	\hline
	Test ID & FR & Inputs & Expected Values & Actual Values & Result \\
	\hline
	US1 & FR-SP-1, FR-SP-2, FR-SP-3 and FR-SP-5  & \{name:'Test Name',email: 'test123@gmail.com', password: '12345'\}  & Status Code = 201 & Status Code = 201 & \textcolor{Green}{PASS} \\
	\hline
	
\end{tabularx}
\caption{User Section Unit Test}
\label{table:workout-unit-tests}
\end{table}

\newpage

\section{Changes Due to Testing}
Formal testing did not reveal any necessary changes in terms of module interfacing, decomposition, or internal design. Changes made to code were to address bugs and logical errors revealed by the testing plan. User interface improvements were made throughout the development process in response to feedback from developers and informal testers.

\section{Automated Testing}
Jest was used to automate the unit tests
\section{Trace to Requirements}
\begin{table}[H]
\begin{center}
	\caption{\textbf{Traceability Matrix for Login Page Functional Requirements}}
	\begin{tabularx}{\textwidth}{cc|c|c|c|c|c|c|c|c|c|c|c|c|}
		\cline{3-11}
		& & \multicolumn{9}{ c|}{Requirements} \\ \cline{3-11}
		& & FR1  & FR2 & FR3 & FR4 & FR5 & FR6 & FR7 & FR8 & FR9 \\ \cline{1-11}
		\multicolumn{1}{ |c| }{\multirow{10}{*}{Test Cases} } &
		\multicolumn{1}{|c| }{ FR-LP-1} &X&&&&&&&& \\ \cline{2-11}
		\multicolumn{1}{|c| }{} 	                  &
		\multicolumn{1}{|c| }{ FR-LP-2} &&X&&&&&&& \\ \cline{2-11}
		\multicolumn{1}{|c| }{} 	                  &
		\multicolumn{1}{|c| }{ FR-LP-3} &&&X&&&&&& \\ \cline{2-11}
		\multicolumn{1}{|c| }{} 	                  &
		\multicolumn{1}{|c| }{ FR-LP-4} &&&&X&&&&& \\ \cline{2-11}
		\multicolumn{1}{|c| }{}                        &
		\multicolumn{1}{|c| }{ FR-LP-5} &&&&&X&&&& \\ \cline{2-11}
		\multicolumn{1}{|c| }{} 	                  &
		\multicolumn{1}{|c| }{ FR-LP-6} &&&&&&X&&& \\ \cline{2-11}
		\multicolumn{1}{|c| }{} 	                  &
		\multicolumn{1}{|c| }{ FR-LP-7} &&&&&&&X&& \\ \cline{2-11}
		\multicolumn{1}{|c| }{}                        &
		\multicolumn{1}{|c| }{ FR-LP-8} &&&&&&&&X& \\ \cline{2-11}
		\multicolumn{1}{|c| }{}                        &
		\multicolumn{1}{|c| }{ FR-LP-9} &&&&&&&&&X \\ \cline{1-11}
	\end{tabularx}
\end{center}
\end{table}

\begin{table}[H]
\begin{center}
	\caption{\textbf{Traceability Matrix for Signup Page Functional Requirements}}
	\begin{tabularx}{\textwidth}{cc|c|c|c|c|c|c|c|c|}
		\cline{3-7}
		& & \multicolumn{5}{ c|}{Requirements} \\ \cline{3-7}
		& & FR10  & FR11 & FR12 & FR13 & FR14 \\ \cline{1-7}
		\multicolumn{1}{ |c| }{\multirow{6}{*}{Test Cases} } &
		\multicolumn{1}{|c| }{ FR-SP-1} &X&&&& \\ \cline{2-7}
		\multicolumn{1}{|c| }{} 	                  &
		\multicolumn{1}{|c| }{ FR-SP-2} &&X&&& \\ \cline{2-7}
		\multicolumn{1}{|c| }{} 	                  &
		\multicolumn{1}{|c| }{ FR-SP-3} &&&X&& \\ \cline{2-7}
		\multicolumn{1}{|c| }{} 	                  &
		\multicolumn{1}{|c| }{ FR-SP-4} &&&&X& \\ \cline{2-7}
		\multicolumn{1}{|c| }{}                        &
		\multicolumn{1}{|c| }{ FR-SP-5} &&&&&X \\ \cline{1-7}
	\end{tabularx}
\end{center}
\end{table}

\begin{table}[H]
\begin{center}
	\caption{\textbf{Traceability Matrix for Main Page Functional Requirements}}
	\begin{tabularx}{\textwidth}{cc|c|c|c|c|c|c|c|c|}
		\cline{3-7}
		& & \multicolumn{4}{ c|}{Requirements} \\ \cline{3-7}
		& & FR15  & FR16 & FR17 & FR18 & FR30 \\ \cline{1-7}
		\multicolumn{1}{ |c| }{\multirow{5}{*}{Test Cases} } &
		\multicolumn{1}{|c| }{ FR-MP-1} &X&&&& \\ \cline{2-7}
		\multicolumn{1}{|c| }{} 	                  &
		\multicolumn{1}{|c| }{ FR-MP-2} &&X&&&X \\ \cline{2-7}
		\multicolumn{1}{|c| }{} 	                  &
		\multicolumn{1}{|c| }{ FR-MP-3} &&&X&& \\ \cline{2-7}
		\multicolumn{1}{|c| }{} 	                  &
		\multicolumn{1}{|c| }{ FR-MP-4} &&&&X& \\ \cline{1-7}
	\end{tabularx}
\end{center}
\end{table}

\begin{table}[H]
\begin{center}
	\caption{\textbf{Traceability Matrix for Diet Page Functional Requirements}}
	\begin{tabularx}{\textwidth}{cc|c|c|c|c|c|c|c|c|c|c|c|c|c|}
		\cline{3-12}
		& & \multicolumn{10}{ c|}{Requirements} \\ \cline{3-12}
		& & FR19  & FR20 & FR21 & FR22 & FR23-25  & FR26 & FR27 & FR28 & FR29 & F30 \\ \cline{1-12}
		\multicolumn{1}{ |c| }{\multirow{11}{*}{Test Cases} } &
		\multicolumn{1}{|c| }{ FR-DS-1} &X&&&&&&&&& \\ \cline{2-12}
		\multicolumn{1}{|c| }{} 	                  &
		\multicolumn{1}{|c| }{ FR-DS-2} &&X&&&&&&&& \\ \cline{2-12}
		\multicolumn{1}{|c| }{} 	                  &
		\multicolumn{1}{|c| }{ FR-DS-3} &&&X&&&&&&& \\ \cline{2-12}
		\multicolumn{1}{|c| }{} 	                  &
		\multicolumn{1}{|c| }{ FR-DS-4} &&&&X&&&&&& \\ \cline{2-12}
		\multicolumn{1}{|c| }{} 	                  &
		\multicolumn{1}{|c| }{ FR-DS-5} &&&&&X&&&&& \\ \cline{2-12}
		\multicolumn{1}{|c| }{} 	                  &
		\multicolumn{1}{|c| }{ FR-DS-6} &&&&&&X&&&& \\ \cline{2-12}
		\multicolumn{1}{|c| }{} 	                  &
		\multicolumn{1}{|c| }{ FR-DS-7} &&&&&&&X&&& \\ \cline{2-12}
		\multicolumn{1}{|c| }{} 	                  &
		\multicolumn{1}{|c| }{ FR-DS-8} &&&&&&&&X&X& \\ \cline{1-12}
	\end{tabularx}
\end{center}
\end{table}

\begin{table}[H]
\begin{center}
	\caption{\textbf{Traceability Matrix for Workout Page Functional Requirements}}
	\begin{tabularx}{\textwidth}{cc|c|c|c|c|c|c|c|c|}
		\cline{3-7}
		& & \multicolumn{5}{ c|}{Requirements} \\ \cline{3-7}
		& & FR31  & FR32 & FR33 & FR34 & FR35 \\ \cline{1-7}
		\multicolumn{1}{ |c| }{\multirow{6}{*}{Test Cases} } &
		\multicolumn{1}{|c| }{ FR-WP-1} &X&&&& \\ \cline{2-7}
		\multicolumn{1}{|c| }{} 	                  &
		\multicolumn{1}{|c| }{ FR-WP-2} &&X&&& \\ \cline{2-7}
		\multicolumn{1}{|c| }{} 	                  &
		\multicolumn{1}{|c| }{ FR-WP-3} &&&X&& \\ \cline{2-7}
		\multicolumn{1}{|c| }{} 	                  &
		\multicolumn{1}{|c| }{ FR-WP-4} &&&&X& \\ \cline{2-7}
		\multicolumn{1}{|c| }{}                        &
		\multicolumn{1}{|c| }{ FR-WP-5} &&&&&X \\ \cline{1-7}
	\end{tabularx}
\end{center}
\end{table}


\begin{table}[H]
\begin{center}
	\caption{\textbf{Traceability Matrix for Rest Section Functional Requirements}}
	\begin{tabularx}{\textwidth}{cc|c|c|c|c|c|}
		\cline{3-4}
		& & \multicolumn{2}{ c|}{Requirements} \\ \cline{3-4}
		& & FR36  & FR37 \\ \cline{1-4}
		\multicolumn{1}{ |c| }{\multirow{3}{*}{Test Cases} } &
		\multicolumn{1}{|c| }{ FR-RS-1} &X& \\ \cline{2-4}
		\multicolumn{1}{|c| }{}                        &
		\multicolumn{1}{|c| }{ FR-RS-2} &&X \\ \cline{1-4}
	\end{tabularx}
\end{center}
\end{table}

\begin{table}[H]
\begin{center}
	\caption{\textbf{Traceability Matrix for Look and Feel Nonfunctional Requirements}}
	\begin{tabularx}{\textwidth}{cc|c|c|c|c|c|}
		\cline{3-4}
		& & \multicolumn{2}{ c|}{Requirements} \\ \cline{3-4}
		& & LF1  & LF2 \\ \cline{1-4}
		\multicolumn{1}{ |c| }{\multirow{3}{*}{Test Cases} } &
		\multicolumn{1}{|c| }{ NFR-LF1} &X& \\ \cline{2-4}
		\multicolumn{1}{|c| }{}                        &
		\multicolumn{1}{|c| }{ NFR-LF22} &&X \\ \cline{1-4}
	\end{tabularx}
\end{center}
\end{table}


\begin{table}[H]
\begin{center}
	\caption{\textbf{Traceability Matrix for Usability and Humanity Nonfunctional Requirements}}
	\begin{tabularx}{\textwidth}{cc|c|c|c|c|c|c|c|c|c|}
		\cline{3-8}
		& & \multicolumn{6}{ c|}{Requirements} \\ \cline{3-8}
		& & UH1  & UH2 & UH3 & UH4 & UH5 & UH6 \\ \cline{1-8}
		\multicolumn{1}{ |c| }{\multirow{6}{*}{Test Cases} } &
		\multicolumn{1}{|c| }{ NFR-UH1} &X&&&&& \\ \cline{2-8}
		\multicolumn{1}{|c| }{} 	                  &
		\multicolumn{1}{|c| }{ NFR-UH2} &&X&&&& \\ \cline{2-8}
		\multicolumn{1}{|c| }{} 	                  &
		\multicolumn{1}{|c| }{ NFR-UH3} &&&X&&& \\ \cline{2-8}
		\multicolumn{1}{|c| }{} 	                  &
		\multicolumn{1}{|c| }{ NFR-UH4} &&&&X&&\\ \cline{2-8}
		\multicolumn{1}{|c| }{}                        &
		\multicolumn{1}{|c| }{NFR-UH5} &&&&&X& \\ \cline{1-8}
	\end{tabularx}
\end{center}
\end{table}

\begin{table}[H]
\begin{center}
	\caption{\textbf{Traceability Matrix for Perfromance Nonfunctional Requirements}}
	\begin{tabularx}{\textwidth}{cc|c|c|c|c|c|c|c|c|}
		\cline{3-7}
		& & \multicolumn{5}{ c|}{Requirements} \\ \cline{3-7}
		& & PE1  & PE2 & PE3 & PE4 & PE5 \\ \cline{1-7}
		\multicolumn{1}{ |c| }{\multirow{6}{*}{Test Cases} } &
		\multicolumn{1}{|c| }{ NFR-PE1} &X&&&& \\ \cline{2-7}
		\multicolumn{1}{|c| }{} 	                  &
		\multicolumn{1}{|c| }{ NFR-PE2} &&X&&& \\ \cline{2-7}
		\multicolumn{1}{|c| }{} 	                  &
		\multicolumn{1}{|c| }{ NFR-PE3} &&&&X& \\ \cline{2-7}
		\multicolumn{1}{|c| }{} 	                  &
		\multicolumn{1}{|c| }{ NFR-PE4} &&&&&X \\ \cline{1-7}
	\end{tabularx}
\end{center}
\end{table}

\begin{table}[H]
\begin{center}
	\caption{\textbf{Traceability Matrix for Operational Nonfunctional Requirements}}
	\begin{tabularx}{\textwidth}{cc|c|c|c|c|c|}
		\cline{3-4}
		& & \multicolumn{2}{ c|}{Requirements} \\ \cline{3-4}
		& & OE1  & OE2 \\ \cline{1-4}
		\multicolumn{1}{ |c| }{\multirow{1}{*}{Test Cases} } &
		\multicolumn{1}{|c| }{ NFR-OE1} &X& \\ \cline{1-4}
	\end{tabularx}
\end{center}
\end{table}

\begin{table}[H]
\begin{center}
	\caption{\textbf{Traceability Matrix for Maintainability and Portability Nonfunctional Requirements}}
	\begin{tabularx}{\textwidth}{cc|c|c|c|c|c|c|}
		\cline{3-5}
		& & \multicolumn{3}{ c|}{Requirements} \\ \cline{3-5}
		& & MP1  & MP2 & MP3 \\ \cline{1-5}
		\multicolumn{1}{ |c| }{\multirow{2}{*}{Test Cases} } &
		\multicolumn{1}{|c| }{ NFR-MP1} &X&& \\ \cline{2-5}
		\multicolumn{1}{|c| }{}                        &
		\multicolumn{1}{|c| }{ NFR-MP2} &&X& \\ \cline{1-5}
	\end{tabularx}
\end{center}
\end{table}

\begin{table}[H]
\begin{center}
	\caption{\textbf{Traceability Matrix for Security Nonfunctional Requirements}}
	\begin{tabularx}{\textwidth}{cc|c|c|c|c|c|}
		\cline{3-4}
		& & \multicolumn{2}{ c|}{Requirements} \\ \cline{3-4}
		& & SE1  & SE2 \\ \cline{1-4}
		\multicolumn{1}{ |c| }{\multirow{3}{*}{Test Cases} } &
		\multicolumn{1}{|c| }{NFR-SE1} &X& \\ \cline{2-4}
		\multicolumn{1}{|c| }{}                        &
		\multicolumn{1}{|c| }{NFR-SE2} &&X \\ \cline{1-4}
	\end{tabularx}
\end{center}
\end{table}

\begin{table}[H]
\begin{center}
	\caption{\textbf{Traceability Matrix for Cultural and Political Nonfunctional Requirements}}
	\begin{tabularx}{\textwidth}{cc|c|c|c|c|}
		\cline{3-3}
		& & \multicolumn{1}{ c|}{Requirements} \\ \cline{3-3}
		& & CU1 \\ \cline{1-3}
		\multicolumn{1}{ |c| }{\multirow{1}{*}{Test Cases} } &
		\multicolumn{1}{|c| }{NFR-CU1} &X \\ \cline{1-3}
	\end{tabularx}
\end{center}
\end{table}

\section{Trace to Modules}		
\begin{table}[H]
\centering
\small
\renewcommand{\arraystretch}{0.9}
\begin{tabular}{p{0.2\textwidth} p{0.6\textwidth}}
	\toprule
	\textbf{Req.} & \textbf{Modules}\\
	\midrule
	FR-LP-1 & M3\\
	FR-LP-2 & M3\\
	FR-LP-3 & M3\\
	FR-LP-4 & M3\\
	FR-LP-5 & M3\\
	FR-LP-6 & M3\\
	FR-LP-7 & M3\\
	FR-LP-8 & M3\\
	FR-LP-9 & M3\\
	FR-SP-1 & M18\\
	FR-SP-2 & M18\\
	FR-SP-3 & M18\\
	FR-SP-4 & M18\\
	FR-SP-5 & M18\\
	FR-MP-1 & M1\\
	FR-MP-2 & M1\\
	FR-MP-3 & M1\\
	FR-MP-4 & M1\\
	FR-DS-1 & M7\\
	FR-DS-2 & M7\\
	FR-DS-3 & M7\\
	FR-DS-4 & M8\\
	FR-DS-5 & M8, M10\\
	FR-DS-6 & M11\\
	FR-DS-7 & M11\\
	FR-DS-8 & M9\\
	FR-WP-1 & M14\\
	FR-WP-2 & M14\\
	FR-WP-3 & M15\\
	FR-WP-4 & M15\\
	FR-WP-5 & M17\\
	FR-RS-1 & M5\\
	FR-RS-2 & M6\\
	\bottomrule
\end{tabular}
\caption{Trace Between Requirements and Modules}
\label{TblRT}
\end{table}
\section{Code Coverage Metrics}
N/A
\section{Reflection Appendix}
Bill Nguyen: for the vnv plan, it was more formulation rather than implementation, we looked at how we were going to test our project rather than actually doing it. For the vnv report it was more the implementation of our formulation where we wrote actual unit/automated tests and tested our project fully and than compared it to our vnv plan to see what requirements etc. did we satisfy and maybe find things we need to improve on.\\\\
Hasan Kibria: In comparison to the vnv plan, the vnv report was more based on practicality an implementation. To complete it fully, there was real code and test cases that had to be thought of an implemented so that they could then be documented in the vnv report. In the vnv plan it was more of an outlook of what we envisioned our testing to look like.\\\\
Syed Bokhari: The VNV plan focuses on formulating the testing approach and strategies, while the VNV report is more concerned with the implementation and documentation of the actual testing process. The VNV report involves the creation and execution of test cases, which are then compared to the plan to identify any gaps or areas for improvement. The VNV plan provides a high-level view of the testing process, while the VNV report is a more detailed account of the actual testing activities.\\\\
Youssef Dahab: Both the VnV plan and VnV report take inspiration from the functional and non-functional requirements in the SRS document. The VnV plan described how we were going to test our functional and non-functional requirements while the VnV report described the results of performing those tests.\\\\
Logan Brown: The VnV plan was more abstract without knowledge of the implementation. The VnV report documents the more refined and directed tests that were performed which could now be completed due to the implementation being more concrete. I have a better idea of how VnV is carried out and the importance of "faking the design process" in the initial stages to make later VnV much easier.\\\\
Mahmoud Anklis: The VnV plan is designed to come up with a testing and verification methodology that would ensure that the software application adheres to the functional and non-functional requirements. On the other hand, the VnV report focuses on the actual execution of the tests which requires implementation steps.\\\\

\bibliographystyle{plainnat}

\bibliography{../../refs/References}

\end{document}