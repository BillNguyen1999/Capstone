\documentclass[12pt, titlepage]{article}

\usepackage{booktabs}
\usepackage{tabularx}
\usepackage{hyperref}
\hypersetup{
colorlinks,
citecolor=black,
filecolor=black,
linkcolor=red,
urlcolor=blue
}
\usepackage[round]{natbib}

\usepackage{titlesec}
\usepackage{placeins}
\usepackage{graphicx}
\usepackage{xkeyval}
\usepackage[dvipsnames]{xcolor} % for different colour comments
\usepackage{tabto}
\usepackage{mdframed}
\usepackage{lscape}
\usepackage{multirow}


%% Comments

\usepackage{color}

\newif\ifcomments\commentstrue %displays comments
%\newif\ifcomments\commentsfalse %so that comments do not display

\ifcomments
\newcommand{\authornote}[3]{\textcolor{#1}{[#3 ---#2]}}
\newcommand{\todo}[1]{\textcolor{red}{[TODO: #1]}}
\else
\newcommand{\authornote}[3]{}
\newcommand{\todo}[1]{}
\fi

\newcommand{\wss}[1]{\authornote{blue}{SS}{#1}} 
\newcommand{\plt}[1]{\authornote{magenta}{TPLT}{#1}} %For explanation of the template
\newcommand{\an}[1]{\authornote{cyan}{Author}{#1}}


\makeatletter
\define@cmdkey      [TP] {test}     {name}       {}
\define@cmdkey      [TP] {test}     {desc}       {}
\define@cmdkey      [TP] {test}     {type}       {}
\define@cmdkey      [TP] {test}     {init}       {}
\define@cmdkey      [TP] {test}     {input}      {}
\define@cmdkey      [TP] {test}     {output}     {}
\define@cmdkey      [TP] {test}     {pass}       {}
\define@cmdkey      [TP] {test}     {user}       {}
\define@cmdkey      [TP] {test}     {result}     {}

\newcounter{TestNum}
\newcommand{\testauto}[1]{
\setkeys[TP]{test}{#1}
\refstepcounter{TestNum}
\begin{mdframed}[linewidth=1pt]
	\begin{tabularx}{\textwidth}{@{}p{3cm}X@{}}
		{\bf Test \#\theTestNum:} & {\bf \cmdTP@test@name}\\[\baselineskip]
		{\bf Description:} & \cmdTP@test@desc\\[0.5\baselineskip]
		{\bf Type:} & \cmdTP@test@type\\[0.5\baselineskip]
		{\bf Initial State:} & \cmdTP@test@init\\[0.5\baselineskip]
		{\bf Input:} & \cmdTP@test@input\\[0.5\baselineskip]
		{\bf Output:} & \cmdTP@test@output\\[0.5\baselineskip]
		{\bf Expected:} & \cmdTP@test@pass\\[\baselineskip]
		{\bf Result:} & \cmdTP@test@result
	\end{tabularx}
\end{mdframed}
}

\newcommand{\testautob}[1]{
\setkeys[TP]{test}{#1}
\refstepcounter{TestNum}
\begin{mdframed}[linewidth=1pt]
	\begin{tabularx}{\textwidth}{@{}p{3cm}X@{}}
		{\bf Test \#\theTestNum:} & {\bf \cmdTP@test@name}\\[\baselineskip]
		{\bf Description:} & \cmdTP@test@desc\\[0.5\baselineskip]
		{\bf Type:} & \cmdTP@test@type\\[0.5\baselineskip]
		{\bf Pass:} & \cmdTP@test@pass\\[\baselineskip]
		{\bf Result:} & \cmdTP@test@result
	\end{tabularx}
\end{mdframed}
}

\newcommand{\testmanual}[1]{
\setkeys[TP]{test}{#1}
\refstepcounter{TestNum}
\begin{mdframed}[linewidth=1pt]
	\begin{tabularx}{\textwidth}{@{}p{3cm}X@{}}
		{\bf Test \#\theTestNum:} & {\bf \cmdTP@test@name}\\[\baselineskip]
		{\bf Description:} & \cmdTP@test@desc\\[0.5\baselineskip]
		{\bf Type:} & \cmdTP@test@type\\[0.5\baselineskip]
		{\bf Tester(s):} & \cmdTP@test@user\\[0.5\baselineskip]
		{\bf Pass:} & \cmdTP@test@pass\\[\baselineskip]
		{\bf Result:} & \cmdTP@test@result
	\end{tabularx}
\end{mdframed}
}

\begin{document}

\title{Verification and Validation Report: REVITALIZE} 
\author{Author Name}
\date{\today}

\maketitle

\pagenumbering{roman}

\section{Revision History}

\begin{tabularx}{\textwidth}{p{3cm}p{2cm}X}
	\toprule {\bf Date} & {\bf Version} & {\bf Notes}\\
	\midrule
	Date 1 & 1.0 & Notes\\
	Date 2 & 1.1 & Notes\\
	\bottomrule
\end{tabularx}

~\newpage

\section{Symbols, Abbreviations and Acronyms}

\renewcommand{\arraystretch}{1.2}
\begin{tabular}{l l} 
	\toprule		
	\textbf{symbol} & \textbf{description}\\
	\midrule 
	T & Test\\
	\bottomrule
\end{tabular}\\

\wss{symbols, abbreviations or acronyms -- you can reference the SRS tables if needed}

\newpage

\tableofcontents

\listoftables %if appropriate

\listoffigures %if appropriate

\newpage

\pagenumbering{arabic}

This document ...

\section{Functional Requirements Evaluation}

\subsection{Login Page}
\testauto{
	name = {FR-LP-1},
	desc =  ?,
	type = Manual,
	init = Loading stage of the login page,
	input = An event that loads the login page,
	output = Login page is displayed with all necessary components,
	pass = ?,
	result = \textcolor{Green}{PASS}
}
\testauto{
	name = {FR-LP-2},
	desc =  ?,
	type = Manual,
	init = Login page is displayed with username textbox,
	input = Enter username information in textbox,
	output = Username information entered is displayed in textbox,
	pass = ?,
	result = \textcolor{Green}{PASS}
}
\testauto{
	name = {FR-LP-3},
	desc =  ?,
	type = Manual,
	init = Login page is displayed with password textbox,
	input = Enter password information in textbox,
	output = Password information entered is displayed in textbox via hidden text,
	pass = ?,
	result = \textcolor{Green}{PASS}
}
\testauto{
	name = {FR-LP-4},
	desc =  ?,
	type = Manual,
	init = Login page is displayed with login button,
	input = Click the login button,
	output = User logs in after the system checks the validity of the input parameters in the login page,
	pass = ?,
	result = \textcolor{Green}{PASS}
}
\testauto{
	name = {FR-LP-5},
	desc =  ?,
	type = Manual,
	init = Login page is displayed with forgot password button,
	input = Click forgot password button,
	output = Display forgot password screen with textbox to enter email,
	pass = ?,
	result = \textcolor{Green}{PASS}
}
\testauto{
	name = {FR-LP-6},
	desc =  ?,
	type = Manual,
	init = Login page is displayed with stay logged in checkbox that is empty,
	input = Click stay logged in checkbox,
	output = Display a check-mark in the stay logged in checkbox if checkbox is empty.
	Else if checkbox contains check-mark already it will then display an empty checkbox,
	pass = ?,
	result = \textcolor{Green}{PASS}
}
\testauto{
	name = {FR-LP-7},
	desc =  ?,
	type = Manual,
	init = Loading stage of REVITALIZE where previous state had stay logged in
	checkbox checked,
	input = An event that loads REVITALIZE,
	output = Display main page, with same data from previous state of main page,
	pass = ?,
	result = \textcolor{Green}{PASS}
}
\testauto{
	name = {FR-LP-8},
	desc =  ?,
	type = Manual,
	init = Login page is displayed with sign up button,
	input = Click sign up button,
	output = Loads and displays sign up page,
	pass = ?,
	result = \textcolor{Green}{PASS}
}
\testauto{
	name = {FR-LP-9},
	desc =  ?,
	type = Manual,
	init = Login page is displayed with inputted information in username and pass-
	word text-boxes,
	input = Click login button,
	output = If failure state, display an invalid password or username banner, else if success
	state, load and display main page,
	pass = ?,
	result = \textcolor{Green}{PASS}
}

\subsection{Signup Page}
\testauto{
	name = {FR-SP-1},
	desc =  ?,
	type = Manual,
	init = Signup page is displayed with username textbox,
	input = Enter username information in textbox,
	output = Username information entered is displayed in textbox,
	pass = ?,
	result = \textcolor{Green}{PASS}
}
\testauto{
	name = {FR-SP-2},
	desc =  ?,
	type = Manual,
	init = Signup page is displayed with password textbox,
	input = Enter password information in textbox,
	output = Password information entered is displayed in textbox via hidden text,
	pass = ?,
	result = \textcolor{Green}{PASS}
}
\testauto{
	name = {FR-SP-3},
	desc =  ?,
	type = Manual,
	init = Signup page is displayed with email textbox,
	input = Enter email information in textbox,
	output = Email information entered is displayed in textbox,
	pass = ?,
	result = \textcolor{Green}{PASS}
}
\testauto{
	name = {FR-SP-4},
	desc =  ?,
	type = Manual,
	init = Signup page is displayed with signup button,
	input = Click the signup button,
	output = User signs up after the system checks the validity of the input parameters on the signup page,
	pass = ?,
	result = \textcolor{Green}{PASS}
}
\testauto{
	name = {FR-SP-5},
	desc =  ?,
	type = Manual,
	init = Signup page is displayed with inputted information in username and
	password text-boxes,
	input = Click signup button,
	output = If failure state then display an invalid username/password or email banner. Else if success state then load and display login page,
	pass = ?,
	result = \textcolor{Green}{PASS}
}

\subsection{Main Page}
\testauto{
	name = {FR-MP-1},
	desc =  ?,
	type = Manual,
	init = Main page is displayed with calender of current date,
	input = An event that loads the main page,
	output = Main page is displayed with all necessary components,
	pass = ?,
	result = \textcolor{Green}{PASS}
}
\testauto{
	name = {FR-MP-2},
	desc =  ?,
	type = Manual,
	init = Main page and Diet\comma{,} Workout\comma{,} Rest sections are displayed with previous day and next day buttons,
	input = An event that loads the main page\comma{,} Diet\comma{,} Workout\comma{,} Rest sections and the previous day and next day buttons are clicked,
	output = Main page\comma{,} Diet\comma{,} Workout\comma{,} Rest sections are displayed with previous day and next day buttons. Once the next day button is clicked\comma{,} the calendar refreshes the calendar information for the next day. Once the previous day button is clicked\comma{,} the calendar refreshes the calendar information for the previous day,
	pass = ?,
	result = \textcolor{Green}{PASS}
}
\testauto{
	name = {FR-MP-3},
	desc =  ?,
	type = Manual,
	init = Each interaction after leaving the main page must have a visible back button,
	input = An event that loads the next user interface after leaving the main page and the back button is clicked,
	output = The next user interface after leaving the main page is displayed with a back button. Once the back button is clicked the main page is loaded,
	pass = ?,
	result = \textcolor{Green}{PASS}
}
\testauto{
	name = {FR-MP-4},
	desc =  ?,
	type = Manual,
	init = Main page is displayed with Diet\comma{,} Exercise and Rest buttons available to click,
	input = An event that loads the main page and the Diet\comma{,} Exercise and Rest buttons are clicked,
	output = Main page is displayed with Diet\comma{,} Exercise and Rest buttons. If the Diet button is clicked\comma{,} the Diet interface is loaded. If the Exercise button is clicked\comma{,} the Exercise interface is loaded. If the Rest button is clicked\comma{,} the Rest interface is loaded,
	pass = ?,
	result = \textcolor{Green}{PASS}
}

\subsection{Diet Section Page}
\testauto{
	name = {FR-DS-1},
	desc =  ?,
	type = Manual,
	init = Diet section is initialized for the first time and an initial information dialog is launched,
	input = An event that loads the diet section for the first time,
	output = A fillable dialog box is launched with height\comma{,} dietary information\comma{,} weight and calorie information,
	pass = ?,
	result = \textcolor{Green}{PASS}
}
\testauto{
	name = {FR-DS-2},
	desc =  ?,
	type = Manual,
	init = Diet section is initialized for the first time and an initial information dialog is launched,
	input = Initial information dialog values are filled,
	output = Initial information values are saved to the database,
	pass = ?,
	result = \textcolor{Green}{PASS}
}
\testauto{
	name = {FR-DS-3},
	desc =  ?,
	type = Manual,
	init = Section is initialized with a list of food logged for the current calendar day,
	input = An event that loads the rest section,
	output = A list of inputted food is loaded for the current calendar day,
	pass = ?,
	result = \textcolor{Green}{PASS}
}
\testauto{
	name = {FR-DS-4},
	desc =  ?,
	type = Manual,
	init = Diet section is displayed with add food button,
	input = Click add food button,
	output = A user interface is launched that lets the user select between searching for food or adding a custom meal,
	pass = ?,
	result = \textcolor{Green}{PASS}
}
\testauto{
	name = {FR-DS-5},
	desc =  ?,
	type = Manual,
	init = Food adding user interface is displayed with search food button,
	input = Click the search food button,
	output = A recipe criteria user interface is launched that displays a list of modifiable criteria and a search button,
	pass = ?,
	result = \textcolor{Green}{PASS}
}
\testauto{
	name = {FR-DS-6},
	desc =  ?,
	type = Manual,
	init = Recipe criteria user interface is launched,
	input = Search criteria is modied and search button is clicked,
	output = List of recipes are loaded correctly based on constraints of search criteria,
	pass = ?,
	result = \textcolor{Green}{PASS}
}
\testauto{
	name = {FR-DS-7},
	desc =  ?,
	type = Manual,
	init = Recipe list is loaded based on search constraints,
	input = Add recipe button is clicked,
	output = Selected recipe is added to the list of food logged on the current calendar day,
	pass = ?,
	result = \textcolor{Green}{PASS}
}
\testauto{
	name = {FR-DS-8},
	desc =  ?,
	type = Manual,
	init = Food adding interface is displayed with add custom meal button,
	input = Click add custom meal button,
	output = A dialog box is launched that lets the user fill in custom meal information. The meal is added to the food log list of the current calendar day,
	pass = ?,
	result = \textcolor{Green}{PASS}
}

\subsection{Workout Section Page}
\testauto{
	name = {FR-WS-1},
	desc =  ?,
	type = Manual,
	init = Workout section is initialized with a preset list of exercises of the current calendar day,
	input = An event that loads the workout section,
	output = A preset list of exercises is loaded for the current calendar day,
	pass = ?,
	result = \textcolor{Green}{PASS}
}
\testauto{
	name = {FR-WS-2},
	desc =  ?,
	type = Manual,
	init = Workout section is displayed with add exercise button,
	input = Click add exercise button,
	output = A dialog box is launched that lets the user ll custom exercise information. The exercise is added to the exercise list of the current calendar day,
	pass = ?,
	result = \textcolor{Green}{PASS}
}
\testauto{
	name = {FR-WS-3},
	desc =  ?,
	type = Manual,
	init = Each exercise in the workout section is displayed with a delete exercise button,
	input = Click delete exercise button,
	output = The exercise is deleted from the exercise list of the current calendar day,
	pass = ?,
	result = \textcolor{Green}{PASS}
}
\testauto{
	name = {FR-WS-4},
	desc =  ?,
	type = Manual,
	init = Each exercise in the workout section is displayed with an edit exercise button,
	input = click edit exercise button,
	output = A fillable dialog box is launched with information of the exercise. Once
	the edit exercise button is clicked the dialog box will close and update the exercise
	information in the list of exercises for the current calendar day,
	pass = ?,
	result = \textcolor{Green}{PASS}
}
\testauto{
	name = {FR-WS-5},
	desc =  ?,
	type = Manual,
	init = Workout section is displayed with list of exercises for current calendar day,
	input = An event that loads the workout section,
	output = If repetition and sets for exercises not logged then dialog box for exercise is launched and the missing repetition and set values are highlighted,
	pass = ?,
	result = \textcolor{Green}{PASS}
}

\subsection{Rest Section Page}
\testauto{
	name = {FR-RS-1},
	desc =  ?,
	type = Manual,
	init = Rest section is initialized with the sleep statistics of the current calendar day,
	input = An event that loads the rest section,
	output = Sleep statistics are loaded for the current calendar day,
	pass = ?,
	result = \textcolor{Green}{PASS}
}
\testauto{
	name = {FR-RS-2},
	desc =  ?,
	type = Manual,
	init = Rest section is initialized with the sleep statistics of the current calendar day,
	input = Alter sleep data,
	output = The sleep data is updated with user changes,
	pass = ?,
	result = \textcolor{Green}{PASS}
}

\section{Nonfunctional Requirements Evaluation}

\subsection{Usability}

\subsection{Performance}

\subsection{etc.}

\section{Comparison to Existing Implementation}	

This section will not be appropriate for every project.

\section{Unit Testing}

\section{Changes Due to Testing}

\section{Automated Testing}

\section{Trace to Requirements}

\section{Trace to Modules}		

\section{Code Coverage Metrics}

\bibliographystyle{plainnat}

\bibliography{../../refs/References}

\end{document}