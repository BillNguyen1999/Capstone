\documentclass{article}

\usepackage{booktabs}
\usepackage{tabularx}

\title{Development Plan\\\progname}

\author{\authname}

\date{}

%% Comments

\usepackage{color}

\newif\ifcomments\commentstrue %displays comments
%\newif\ifcomments\commentsfalse %so that comments do not display

\ifcomments
\newcommand{\authornote}[3]{\textcolor{#1}{[#3 ---#2]}}
\newcommand{\todo}[1]{\textcolor{red}{[TODO: #1]}}
\else
\newcommand{\authornote}[3]{}
\newcommand{\todo}[1]{}
\fi

\newcommand{\wss}[1]{\authornote{blue}{SS}{#1}} 
\newcommand{\plt}[1]{\authornote{magenta}{TPLT}{#1}} %For explanation of the template
\newcommand{\an}[1]{\authornote{cyan}{Author}{#1}}

%% Common Parts

\newcommand{\progname}{REVITALIZE} % PUT YOUR PROGRAM NAME HERE
\newcommand{\authname}{Team 13, REVITALIZE
\\ Bill Nguyen
\\ Syed Bokhari
\\ Hasan Kibria
\\ Youssef Dahab
\\ Logan Brown
\\ Mahmoud Anklis} % AUTHOR NAMES                  

\usepackage{hyperref}
    \hypersetup{colorlinks=true, linkcolor=blue, citecolor=blue, filecolor=blue,
                urlcolor=blue, unicode=false}
    \urlstyle{same}
                                


\begin{document}

\begin{table}[hp]
	\caption{Revision History} \label{TblRevisionHistory}
	\begin{tabularx}{\textwidth}{llX}
		\toprule
		\textbf{Date} & \textbf{Developer(s)} & \textbf{Change}\\
		\midrule
		Date1 & Name(s) & Description of changes\\
		Date2 & Name(s) & Description of changes\\
		... & ... & ...\\
		\bottomrule
	\end{tabularx}
\end{table}

\newpage

\maketitle

\wss{Put your introductory blurb here.}

\section{Team Meeting Plan}

\section{Team Communication Plan}

\section{Team Member Roles}

\section{Workflow Plan}
The main repository is named REVITALIZE. The implementation will follow the feature-branch model. Feature development will take place in a branch other than the main branch. This encapsulation makes it easy to work on a feature without disturbing the main codebase. That way, the main branch will never contain broken code. The feature-branch model makes it simple for team members to initiate discussions around a branch by making pull requests to comment on each other’s work.\\\\
The documentation will follow the centralized model. The main branch will serve as a point of entry for all changes to the documentation. All changes will be committed to this branch.\\\\
To report bugs and request modifications to implementation or documentation, GitHub issues will be created and assigned to people working on a specific problem. This will also serve as a way of holding team members accountable for the issues they have to tackle in the future. Labels will be used to classify issues and pull requests to help create a standard workflow in the used repositories.\\\\
If required, milestones will be created to track progress on groups of issues or pull requests to better manage the project through viewing milestones’ view dates, completion percentages, and lists of open and closed issues and pull requests associated with the milestones. Using milestones could become a necessity as project size and complexity increase with time.


\section{Proof of Concept Demonstration Plan}

\subsection{Potential Risks and Difficulties}

\subsubsection{Human Computer Interface and User Experience}
As there are already existing implementations that are similar to this project, the team has to determine how to deliver this project to it's stakeholders in a better way than the other existing applications. This entails improving the human computer interface and user experience aspects of the project. Determining how to tackle this issue entails answering many of the deepest questions and most important questions in computing such as how to design the project to achieve the optimal functionality, learnability, performance, error rates, trustability, retention, and accessibility all at the same time? What does it mean for the interface to be better at all? Easier to use, faster to get tasks done, or less mistakes made by user? Not answering such questions with accuracy can potentially be a blocker to how the team plans to deliver the project to it's stakeholders.

\subsubsection{Design and Implementation}
Some of the team members are not very familiar with the client-server architecture model or with the programming languages and application programming interfaces that the team plans to use. There will be a learning curve that team members are required to get over quickly to deliver the project on time.

\subsubsection{Testing}
The tests, that the team requires to have a working deliverable, entail testing for functionality, usability, interface, performance, and user experience. However, the main concern regarding testing the project is determining how to come up with an effective test plan that tests all the different aspects of the project.

\subsubsection{Scope}
The team does not currently envision the scope to be a risk as the project seems doable as it is. However, further down the line, as team members are designing, documenting, implementing and testing the project, they will be able to determine if the project size is too large and if project goals have to be modified. Thus, project scope is potentially a future risk.

\subsubsection{Future Goal Changes}
The project seems reasonable the way it is planned out. However, from previous years of experience with software development, team members knows that future issues will arise, that currently have been missed or were not thought of or known of, that might force the team to change it's goals or original plan of design and implementation.

\subsubsection{Libraries}
The team does not anticipate any concerns with library installations as standardized frameworks will be used.

\subsubsection{Hardware}
There are no hardware risks or concerns as this is purely a software project.

\subsection{Plan to Overcome Risks}


\section{Technology}

\begin{itemize}
	\item Specific programming language
	\item Specific linter tool (if appropriate)
	\item Specific unit testing framework
	\item Investigation of code coverage measuring tools
	\item Specific plans for Continuous Integration (CI), or an explanation that CI
	is not being done
	\item Specific performance measuring tools (like Valgrind), if
	appropriate
	\item Libraries you will likely be using?
	\item Tools you will likely be using?
\end{itemize}

\section{Coding Standard}

\section{Project Scheduling}
Gantt chart will be used for project scheduling, showing key deliverables and progress. Will be updated regularly throughout the project:
\url{https://github.com/BillNguyen1999/REVITALIZE/tree/main/projectschedule/REVITALIZE.pdf}.

\end{document}