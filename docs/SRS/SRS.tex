\documentclass[12pt,letterpaper]{article}
\usepackage[utf8]{inputenc}
\usepackage[margin=1in]{geometry}

\usepackage{booktabs}
\usepackage{tabularx}
\usepackage{float}
\usepackage{hyperref}
\usepackage{indentfirst}
\usepackage{enumerate}
\usepackage[shortlabels]{enumitem}
\usepackage{xcolor}

\title{System Requirements Specification\\\progname}

\author{\authname}

\date{}

%% Comments

\usepackage{color}

\newif\ifcomments\commentstrue %displays comments
%\newif\ifcomments\commentsfalse %so that comments do not display

\ifcomments
\newcommand{\authornote}[3]{\textcolor{#1}{[#3 ---#2]}}
\newcommand{\todo}[1]{\textcolor{red}{[TODO: #1]}}
\else
\newcommand{\authornote}[3]{}
\newcommand{\todo}[1]{}
\fi

\newcommand{\wss}[1]{\authornote{blue}{SS}{#1}} 
\newcommand{\plt}[1]{\authornote{magenta}{TPLT}{#1}} %For explanation of the template
\newcommand{\an}[1]{\authornote{cyan}{Author}{#1}}

%% Common Parts

\newcommand{\progname}{REVITALIZE} % PUT YOUR PROGRAM NAME HERE
\newcommand{\authname}{Team 13, REVITALIZE
\\ Bill Nguyen
\\ Syed Bokhari
\\ Hasan Kibria
\\ Youssef Dahab
\\ Logan Brown
\\ Mahmoud Anklis} % AUTHOR NAMES                  

\usepackage{hyperref}
    \hypersetup{colorlinks=true, linkcolor=blue, citecolor=blue, filecolor=blue,
                urlcolor=blue, unicode=false}
    \urlstyle{same}
                                


\begin{document}

\maketitle

\begin{table}[hp]
	\caption{Revision History} \label{TblRevisionHistory}
	\begin{tabularx}{\textwidth}{llX}
		\toprule
		\textbf{Date} & \textbf{Developer(s)} & \textbf{Change}\\
		\midrule
		September 29th, 2022 & Youssef Dahab & Project Drivers \\
		October 1st, 2022 & Youssef Dahab & Added Goals of the Project \\
		\bottomrule
	\end{tabularx}
\end{table}

\newpage
\tableofcontents
\newpage

\section{Project Drivers}

\subsection{The Purpose of the Project}
Sustaining a healthy lifestyle requires people to keep track of their eating, exercising, and sleeping habits. This can prove to be a daunting and time consuming thing to do especially when most people are very busy with their lives. The purpose of this project to create an all in one health and wellness mobile application that allows users to manage their diet, exercise, and sleep. REVITALIZE is designed to supply users with the means to improve their health by providing them with meal recipe's based on their nutritional preferences, a personalized workouts planner and a sleep tracker. 

\subsection{Scope}
REVITALIZE will allow users to find recipes for meals based on nutritional preferences such as calories per meal, diet selections, allergies to avoid and ingredients included. The user will be able to count their calorie and nutrient intake through the nutrition logger. The workout planner will allow users to choose from an already existing list of workouts or construct their own workout schedule along with weights, sets, and repetitions. The sleep tracker will provide users with information regarding their sleep. There will be a focus on improving user experience throughout the application along with supplying users with accurate data regarding their health.

\subsection{Goals of the Project}
The goal of this project is to make REVITALIZE, for it's stakeholders, the go to, easy to use, quick, and accessible all in one mobile application for effectively and efficiently managing a person's diet, exercise, and sleep to improve their overall health and well being. The goal of making REVITALIZE a mobile application is for it to be easily accessible to users from their phone at any time and place. Users do not have to memorize their health goals or write them down on a piece of paper that they carry with them all the time. The goal of documenting this project is for stakeholders to have a physical system documentation of a functional product that they can refer to when needed. Stakeholders will be able to match the application to the documentation.

\subsection{The Stakeholders}

\subsubsection{Primary Stakeholders}
Adults and teenagers who want to improve and keep track of their overall health and wellness via an easy to use, all in one application.

\subsubsection{Secondary Stakeholders}
Individuals who may not use the application directly for themselves or are not directly involved with the use of the application but have an indirect benefit. For instance, personal trainers can use REVITALIZE to keep track of workouts, sleep, and the overall health of their clients.

\subsubsection{Facilitating Stakeholders}
Team 13 members building the REVITALIZE application along with Dr.Spencer Smith and the 4G06 TAs.

\section{Project Constraints}

\section{Context Diagrams}

\section{Functional Decomposition Diagrams}

\subsection{Use Case Diagram}

\subsection{Activity Diagram}

\section{Functional Requirements}
\begin{enumerate}[{BE}1.] 
\item The user launches the application
\begin{enumerate}[{FR}1.]
	\item  The system must display a login page upon the start of the application.
	\item  The login page must display fillable username and password  textboxes
	\item  The login page must display a login button
	\item  The login page must display a forgot password button
	\item  The login page must display a stay logged in checkbox
	\item  The system must save prior login information if the stayed logged in checkbox is checked
	\item  The login page must display a sign up button that redirects to a signup page
	\item  The system msut check the validity of the iput parameters in the login page
\end{enumerate}

\item The user selects the sign up button
\begin{enumerate}[resume*]
	\item  The signup page must display fillable username, password, email textboxes
	\item  The signup page must display a signup button
	\item  The system must check the validity of the input parameters in the signup page
\end{enumerate}

\item The user enters the main page after successful login
\begin{enumerate}[resume*]
	\item  The system must display a calender with the current date on successful login
	\item  The system must have a previous day and next day button on each page after successful login
	\item  The system must display a back button on each user interface after a section is selected
	\item  The system must display the sections Diet,Excercise and Rest on the current calender day
\end{enumerate}

\item The user enters the Diet section
\begin{enumerate}[resume*]
	\item  The system must prompt the user to height, input dietery, weight,  calorie information on initial launch of Diet section
	\item  The system must save initial user height, dietery, wieght, calorie information
	\item  The Diet section must initialize with a list of food logged on the current calender day
	\item  The Diet section must display an add food button
	\item  The Diet section must display a search food button
	\item  The search food button must launch a recipe criteria user interface
	\item  The recipe criteria user interface must display a list of modifiable criteria and a search button
	\item  The recipe search must display correct recipe values based on constraints
	\item  The recipe search must display a select recipe and add recipe button
	\item  The Diet section must have an add custom meal button
	\item  The add custom meal button must have fillable textboxes for recipe information
	\item  The previous day and next day button must launch the previous or next calender entry of the user section
\end{enumerate}

\item The user enters the Workout section
\begin{enumerate}[resume*]
	\item  The Workout section must initialize with a preset list of excercises on the current calender day
	\item  The Workout section must have an add excercise and delete excercise button
	\item  The excercises must display an edit excercise button that launches the changeable excercise information when clicked
	\item  The Workout section must have an add excercise and delete excercise button
	\item  The Workout section must prompt the user to add repititions and sets of each excercise logged in the current calender day
\end{enumerate}

\item The user enters the Rest section
\begin{enumerate}[resume*]
	\item  The Rest section must launch with the sleep statistics of the current calender day
	\item  The system must allow the user alter innacurate sleep data
\end{enumerate}
\end{enumerate}

\section{Non-functional Requirements}
\noindent Note: followed the volere requirements template

\subsection{Look and Feel Requirements}
\subsubsection{Appearance Requirements}
\begin{enumerate}[{LF}1.] 
	\item The application must have a neat and attractive design.\\
	{\textbf{Fit Criterion:} A focus group of primary stakeholders such as teenagers and young adults will look at UI/UX design of application and would require an 85\% approval rating.}
\end{enumerate}

\subsubsection{Style Requirements}
\begin{enumerate}[resume*]  
	\item The application must use colours that are appealing and contrasting to make it more accessible and non-intrusive.\\
	{\textbf{Fit Criterion:} A focus group of primary stakeholders such as teenagers and young adults will test application with a focus on colour and need an 85\% approval rating that the associated colours do not intrude/distract users from overall application.}
\end{enumerate}

\subsection{Usability and Humanity Requirements}
\subsubsection{Ease of Use Requirements}
\begin{enumerate}[{UH}1.] 
	\item All aspects and features of mobile application can be used using only one hand/one finger.\\
	{\textbf{Fit Criterion:} 95\% of stakeholders with varying size hands/fingers are able to use all aspects of mobile application using one hand/one finger.}
\end{enumerate}
\begin{enumerate}[resume*]  
	\item The application home page must be simple so that user can access any feature of application in under 10 seconds\\
	{\textbf{Fit Criterion:} 90\% of stakeholders can navigate to any of application features from home page in under 10 seconds. }
	\item The application should be easy to use for targeted demographic\\
	{\textbf{Fit Criterion:} A focus group of primary stakeholders such as teenagers and young adults with youngest age being 14 will test application and need an 85\% approval rating that application was easy to use. }
\end{enumerate}

\subsubsection{Personalization and Internationalization Requirements}
\noindent NOT AVAILABLE

\subsubsection{Learning Requirements}
\begin{enumerate}[resume*] 
	\item Users without any prior experience should be able to use and understand application within 3 iterations of each feature.\\
	{\textbf{Fit Criterion:} 85\% of stakeholders can use and understand basic/common aspects of all features within 3 iterations.}
\end{enumerate}

\subsubsection{Understandability and Politeness Requirements}
\begin{enumerate}[resume*] 
	\item Associated UI aspects such as buttons, drop-downs, words etc. must be consistent\\
	{\textbf{Fit Criterion:} 85\% of stakeholders agree that all UI aspects are simple, consistent and understandable.}
\end{enumerate}

\subsubsection{Accessibility Requirements}
\begin{enumerate}[resume*] 
	\item Mobile application should be compatible with android screen readers tool, for potential users with impaired vision.\\
	{\textbf{Fit Criterion:} Accessibility tests, will be conducted and 95\% of application UI should be able to be read using an android screen reader tool.}
\end{enumerate}

\subsection{Performance Requirements}
\subsubsection{Speed and Latency Requirements}
\begin{enumerate}[{PE}1.] 
	\item All output data of application must take 5 seconds or less to load, based on associated inputs.\\
	{\textbf{Fit Criterion:} Developers will run performance tests and ensure all output data loads within 5 seconds or less for 95\% of all API responses and outputs.}
\end{enumerate}

\subsubsection{Safety-Critical Requirements}
\noindent NOT AVAILABLE

\subsubsection{Precision or Accuracy Requirements}
\begin{enumerate}[resume*] 
	\item All output data/numbers should be accurate for double precision floating points.\\
	{\textbf{Fit Criterion:} Perform associated testing (ex. unit testing) to ensure output is accurate for double precision and passes all test cases. }
\end{enumerate}

\subsubsection{Reliability and Availability Requirements}
\begin{enumerate}[resume*] 
	\item Application must have an uptime of 99\%.\\
	{\textbf{Fit Criterion:} Description provides all necessary information. }
\end{enumerate}

\subsubsection{Robustness or Fault-Tolerance Requirements}
\noindent NOT AVAILABLE

\subsubsection{Capacity Requirements}
\begin{enumerate}[resume*] 
	\item Application can be used by a large amounts of users simultaneously.\\
	{\textbf{Fit Criterion:} Application can withstand the usage of at least 50+ users without performance being affected. }
	\item Application can store/save large amount of data.\\
	{\textbf{Fit Criterion:} Application can store/save 1 million+ of data points for all users. }
\end{enumerate}

\section{Project Issues}

\end{document}